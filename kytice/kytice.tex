\documentclass{article}
\usepackage{fullpage}
\usepackage[czech]{babel}
\usepackage{amsfonts}
\usepackage{hyperref}

\title{\vspace{-2cm}Kytice (Karel Jaromír Erben)\vspace{-2cm}}
\date{}
\author{}

\begin{document}
\maketitle
\section{První část.}
\begin{description}
    \setlength\itemsep{0.15em}
    \item[druh:] lyricko-epický
    \item[žánr:] poezie: balady, některé mají prvky pohádky (Kolovrat), legendy (Libuše), hororu (Svatební košile), pověsti (Poklad)
    \item[téma:] převyprávěná ÚLS, lidové příběhy, nadpřirozené bytosti
    \item[motivy všech:] matka a dcera, láska a smrt, hříchy, růženec, kolovrat, nadpřirozené bytosti, ÚLS, křesťanský pohled na svět, vina a trest, přeměna na přírodní prvky
    \begin{enumerate}
        \vspace{-0.5em}
        \setlength\itemsep{0.15em}
        \item Kytice -- úvod do stejnojmenné sbírky; smrt, vlastenectví
        \item Poklad -- pomíjivost a nepotřebnost bohatství; Žena o Velikonocích najde ve skále poklad, jak ho odnáší, nechá tam své dítě, poté se skála uzavře, zlato se promění v kamení, dítě zůstane ve skále. Matka ho najde o dalších Velikonocích na stejném místě, tentokrát však po pokladu netouží.
        \item Svatební košile -- jak člověka změní smrt, důležitost víry v Boha; Dívka si přeje za každou cenu vidět svého milence, ten je však mrtvý, dovede ji k sobě domů -- na hřbitov, dívka se zachrání jen proto, že se modlí.
        \item Polednice -- nikdy si nepřát něco špatného, trpělivost s dětmi; Přepracovaná matka omylem řekne, ať si pro její ratolest přijde Polednice, což se stane. Matka nakonec dítě uškrtí svým pevným sevřením, protože o něj nechce přijít (ironické, že).
        \item Zlatý kolovrat -- odplata za špatné činy, spravedlnost; Král si chce vzít nevlastní dceru matky, ta však chce provdat svoji vlastní dceru. Tak se taky stane, přičemž nevlastní sestře uřežou nohy, ruce a vypíchnou oči. Nějaký starý pán poté prodá matce s dcerou zlatý kolovrat za části těla, které uřezali, načež je spojí s trupem léčivou vodou a tělo oživne. Matka s dcerou jsou potrestány stejným činem, jako který samy vykonaly.
        \item Štědrý den -- kontrast mezi náhlou radostí a úmrtím; Dvě krásné dívky se na Štědrý den jdou podívat do jezera, kde mají vidět svého budoucího milého, jenže jedna z nich vidí hřbitov $\Rightarrow$ jedna se šťastně vdá, druhá umře.
        \item Holoubek -- Vdova si vzala muže, holoubek (symbol mrtvého muže, kterého zabila) ji přinutí, aby se zabila.
        \item Záhořovo lože -- nikdy není pozdě navrátit se k víře
        \item Vodník -- krvelačná stvůra, konflikt mezi zachráněním sebe nebo svého dítěte; Divka jde k jezeru (proti vůli své matky), spadne do něj, má dítě s vodníkem, ten ji poté na den pustí na zem za její matkou. Dívka se však nechce vrátit, spolu s matkou chtějí, aby jim vodník přinesl její dítě. To se stane, jenže dítě je vejpůl. Každá osoba se zde nějak provinila (třeba dívka: jde k jezeru i když jí matka říká, že nemá; nevrátí se vodníkovi)
        \item Vrba -- Muž pokácí vrbu, do které se vtěluje jeho manželka, čímž ji zabije
        \item Lilie -- Umřela dívka, vtělí se do lilie, muž ji utrhne, na noc se z ní stává opět dívka, chlapec odjede do světa, jeho matka postaví zeď, která stíní lilii, čímž ji zabije.
        \item Dceřina kletba -- dcera má dítě, otec ji zradí, zabije jej, řekne matce, že ji špatně vychovala a tudíž je to její vina, pověsí se
        \item Libuše -- proroctví kněžny Libuše
    \end{enumerate}
Dále se zabýváme rozebíranou básní Kolovrat.
    \item[motivy:] kolovrat, kůň, les, léčivá voda, zlato, chamtivost, sekera, nůž, keř, krev, nohy, ruce, oči
    \item[zařazení výňatku do kontextu díla:] začátek básně, expozice situace, zjistíme, že král si chce vzít dívku, kterou právě potkal
    \item[časoprostor:] ?
    \item[kompoziční výstavba:] 6 částí, chronologicky
\end{description}
\section{Druhá část.}
\begin{description}
    \setlength\itemsep{0.15em}
    \item[vypravěč:] není
    \item[vyprávěcí způsopby:] přímá řeč, \textit{--er} forma
    \item[typy promluv:] přímá řeč
    \item[jazyková stránka:] archaický jazyk, častá zvukomalba
    \item[veršová výstavba:] verš vázaný, trochej / daktyl, rým sdružený
    \item[postavy:]
        \begin{description}
            \setlength\itemsep{0.15em}
            \item[matka a dcera] lakomé, chtějí to nejlepší pro sebe, nebojí se vraždit, za zlato prodají cizincům klidně i části lidského těla, spravedlivě potrestány
            \item[Dora] nevlastní dcera, skromná, šikovná, pracovitá, poslušná, pohledná, usmrcena a později oživena
        \end{description}
    \item[názor:] Kytice pro mě byla daleko čtenářsky stravitelnější než Máj, četbu jsem si více užil, protože jsem u ní nemusel tolik přemýšlet. Mám rád tento typ (mnohdy až hororových) příběhů.
    \item[zdroje:] $ $
    \begin{itemize}
        \setlength\itemsep{0em}
        \item[$-$] ERBEN, Karel Jaromír. \textit{Kytice.} Vydání 3. Praha: SPN, 1973.
    \end{itemize}
\end{description}
\end{document}
