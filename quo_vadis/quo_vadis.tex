\documentclass{article}
\usepackage{fullpage}
\usepackage[czech]{babel}
\usepackage{amsfonts}
\usepackage{hyperref}

\title{\vspace{-2cm}Quo vadis? (Henryk Sienkiewicz)\vspace{-2cm}}
\date{}
\author{}

\begin{document}
\maketitle
\section{První část.}
\begin{description}
    \setlength\itemsep{0.15em}
    \item[druh:] epika
    \item[žánr:] historický román (rozsáhlý děj s odbočkami, vedlejší dějové linky)
    \item[téma:] láska mezi Vinitiem (patricius) a Lygií (otrokyně), nástup křesťanství v pohanském Římě, problémy absolutismu
    \item[motivy:] katakomby, Ježíš, kříž, gladiátorské zápasy, venkov, básně
    \item[zařazení výňatku do kontextu díla:] Po tom, co Nero zapálí Řím (aby měl inspiraci pro svoje dílo, ve kterém chce popisovat požár Troji), je římský lid právem znechucen. Nero tuto situaci musí nějak vyřešit. Jeho rádce Tigellinus navrhne, že by to mohl svést na křesťany. V té době je to totiž jen \uv{malo početná sekta} a nikdo vlastně neví, co jsou nebo o co stojí. Jejich náboženství je zakázané, musí se scházet v katakombách, což je samo o sobě krapet podezřelé. Petronius s tím nesouhlasí a snaží se ho přesvědčit, aby to svedl na někoho jiného. To se mu nepovede, pochopí, že prohrál. Je poslán na venkov a odsouzen k smrti. Proto uspořádá svou poslední hostinu, které jsou oblastí proslulé, kde se sejdou patriciové z celého okolí. Na ní přečte dopis Neronovi (v úryvku) a posléze si podřezá žíly.
    \item[časoprostor:] Řím za doby císaře Nerona, 64 n. l.
    \item[kompoziční výstavba:] řazeno do kapitol, chronologicky s odbočkami
\end{description}
\section{Druhá část.}
\begin{description}
    \setlength\itemsep{0.15em}
    \item[vypravěč:] není
    \item[vyprávěcí způsoby:] \textit{--er} forma
    \item[typy promluv:] přímá a nepřímá řeč, vypravování
    \item[jazyková stránka:] spisovná, někdy zastaralá čeština, někdy vulgární výrazy, přirovnání, ustálené obraty
    \item[postavy:]
        \begin{description}
            \setlength\itemsep{0.15em}
            \item[Vinitius,] patricius a synovec Petronia, rozmazlený, vždy se mu dostane, co chce, má známosti, zamiluje se do Lygie, původně pohan, později konvertuje ke křesťanství
            \item[Lygie,] otrokyně rodu Aulů, křesťanka, musí se skrývat kvůli tomu skrývat, hodná, hezká
            \item[Petronius,] strýc Vinitiův a jeden z poradců Nerona, \textit{arbiter elegantiare}, dobrý básník a poradce ve věcech vkusu; Nero je jeho přítel, on však nemá rád jeho umění -- i tak ho oslavuje $\Rightarrow$ vtíravý, moudrý, na konci spáchá sebevraždu
            \item[Nero,] císař římský, nezáleží mu na své říši, chce tvořit umění, moc mu to však nejde, blázen (nechá zapálit Řím jenom kvůli inspiraci pro svou báseň), mstivý
            \item[Ursus,] silný strážce Lygie
        \end{description}
    \item[názor:] Kniha se mi nelíbila. Je neskutečně zdlouhavá a popisy jsem přeskakoval. Je to první kniha z povinné četby, kterou mě nebavilo číst.
    \item[zdroje:] $ $
    \begin{itemize}
        \setlength\itemsep{0em}
        \item[$-$] SIENKIEWICZ, Henryk. Quo vadis?. Přeložil Václav KREDBA, ilustroval Jan STYKA. Praha: Dobrovský, 2013. Omega (Dobrovský). ISBN 978-80-7390-001-4.
    \end{itemize}
\end{description}
\end{document}
