\documentclass{article}
\usepackage{fullpage}
\usepackage[czech]{babel}
\usepackage{amsfonts}
\usepackage{hyperref}

\title{\vspace{-2cm}Máj (Karel Hynek Mácha)\vspace{-2cm}}
\date{}
\author{}

\begin{document}
\maketitle
\section{První část.}
\begin{description}
    \setlength\itemsep{0.15em}
    \item[druh:] lyricko-epické
    \item[žánr:] poezie: poema
    \item[téma:] láska, smrt vlastního otce, oslava přírody, pomíjivost lidského života
    \item[motivy:] jezero, vězení, měsíc, májová krajina, loď, láska k vlasti, šibenice, kapky
    \item[zařazení výňatku do kontextu díla:] ze začtku díla těstě po expozici prostředí
    \item[časoprostor:] okolí Bezdězu, Máchovo jezero
    \item[kompoziční výstavba:] kompozice chronologická, předzpěv (nijak nesouvisí se zbytkem díla), 4 zpěvy a dvě intermezza (napodobuje hudební terminologii)
    \begin{itemize}
        \vspace{-0.5em}
        \setlength\itemsep{0.15em}
        \item[1. zpěv] dozvídáme se o tom, že Vilém zabil svého otce, protože sváděl Jarmilu, Vilém však nevěděl, že se jedná o jeho otce, Jarmila spáchá sebevraždu, protože cití vinu nad tím, že Vilém je za vraždu odsouzen, jeden z loupežíků (Vilém je jakýsi vůdce) přijíždí na lodi a vše jí vyčítá
        \item[2. zpěv] Vilém čeká na svoji popravu v šatlavě, nepřijímá svoji vinu, z jeho přemítání se dozvídáme, že nevěří na posmrtný život (je ateista), je z celé situace taky vyděšen, protože si uvědomuje, že nastane konex jeho existence
        \item[1. intermezzo] příroda se baví o tom, jak uspořádá Vilémův pohřeb
        \item[3. zpěv] poprava Viléma, líčení májové krajiny -- kontrast mezi jeho utrpením a pěknou přírodou; loučí se se zemí, kterou má tak rád
        \item[2. intermezzo] členové loupežnické bandy truchlí nad svým vůdcem
        \item[4. zpěv] po sedmi letech, promlouvá Hynek (autor), sympatizuje s Vilémem, je mu líto jeho smrti, báseň končí slovy: \uv{Hynku! Viléme!! Jarmilo!!!} -- autor se ztotožňuje s postavami, možná tuší, že ho brzy čeká jeho vlastní smrt, což se nakonec prokáže jako realita
    \end{itemize}
\end{description}
\section{Druhá část.}
\begin{description}
    \setlength\itemsep{0.15em}
    \item[vypravěč:] sám autor
    \item[vyprávěcí způsopby:] \textit{--er} forma, monology, přemítání postav
    \item[typy promluv:] přímá řeč
    \item[jazyková stránka:] archaismy, inverze, mnoho metafor a metonymií, častý paralelismus nebo opakování motivů z dřívějška, epizeuxy, personifikace přírody
    \item[veršová výstavba:] verš vázaný, trochej, rým střídavý
    \item[postavy:]
        \begin{description}
            \setlength\itemsep{0.15em}
            \item[Vilém,] arciloupežník a milenec Jarmily, též vrah svého otce, za svůj hrůzný čin je popraven, ateista, přemítá o životě
            \item[Jarmila,] milenka Vilémova, čekala na něj u jezera, místo něj přijel jeden z loupežníků oznamující jí jeho odsouzení a budoucí popravu, vyčítá jí tuto skutečnost, protože ho podváděla s jeho otcem; kdyby to neudělala, Vilém by ho nezabil a nebyl by odsouzen
            \item[Hynek,] autor, později se vydává na místo příběhu
        \end{description}
    \item[názor:] I když se mi poema líbila, byla pro mě složitá na pochopení. Při čtení mi nedošly veškeré souvislosti. Jakmile jsme je však rozebrali ve škole, vše začalo dávat smysl.
    \item[zdroje:] $ $
    \begin{itemize}
        \setlength\itemsep{0em}
        \item[$-$] MÁCHA, Karel Hynek. \textit{Máj.} Bratislava: Tatran, 1980.
    \end{itemize}
\end{description}
\end{document}
