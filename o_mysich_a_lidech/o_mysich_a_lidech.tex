\documentclass{article}
\usepackage{fullpage}
\usepackage[czech]{babel}
\usepackage{amsfonts}
\usepackage{hyperref}

\title{\vspace{-2cm}O myších a lidech (John Steinbeck)\vspace{-2cm}}
\date{}
\author{}

\begin{document}
\maketitle
\section{První část.}
\begin{description}
    \setlength\itemsep{0.15em}
    \item[druh:] epika (jasný děj, psáno v próze)
    \item[žánr:] novela (kratšího rozsahu, jeden hlavní děj -- i když s retrospektivními odbočkami, na konci výrazná pointa v tom, že George zabil Lennieho -- smrt je pro Lennieho jakýmsi vysvobozením, protože by ho Curley lynčoval, což by bylo daleko horší)
    \item[téma:] mezilidské vztahy, soužití dvou naprosto odlišných lidí, americký sen, život námezních dělníků v daném období
    \item[motivy:] myši, psi a všelijaká domácí zvířata, vrbičky, ranč, fazole, pistole, vlasy smotané do buřtíků, sláma, rasismus, cesta za snem, násilí, hebkost
    \item[zařazení výňatku do kontextu díla:] ze začátku díla, teprve se seznamujeme s postavami, George a lennie putují na ranč, kde budou pracovat, výňatek zachycuje zpomalenost Lennieho, výrazným motivem je myš
    \item[časoprostor:] Amerika (Kalifornie u města Soledad -- \textit{samota}) za třicátých let dvacátého století (hospodářská krize)
    \item[kompoziční výstavba:] děj chronologický s výraznými retrospektivními vzpomínkami (o tetě KLáře, jak se potkali apod.), princip paralelismu (mezi zabitím psa a na konci Lennieho -- oba byli zqbití svým majitelem), popisováno jako ve scénáři, ale děj se posunuje jen formou dialogů, gradace (mrtvá myš, štěně a nakonec člověk), šest kapitol označených čísly, děj trvá asi tři dny
\end{description}
\section{Druhá část.}
\begin{description}
    \setlength\itemsep{0.15em}
    \item[vypravěč:] nezaujatý, nad postavami, jako jediný mluví spisovně
    \item[vyprávěcí způsoby:] \textit{--er forma}, v dialozích \textit{--ich forma}
    \item[typy promluv:] přímá řeč, uvozovací věty, pohledy do myšlenek postav se nevyskytují, míjivé dialogy při svěřování Lenniemu
    \item[jazyková stránka:] každá postava mluví dialektem, jediný vypravěč spisovně
    \item[postavy:]
        \begin{description}
            \setlength\itemsep{0.15em}
            \item[George Milton,] přítel Lennieho, obětavě se o něj stará, protože ho o to údajně požádala Lennieho teta Klára (která už pravděpodobně nežije), ale často je na něj naštvaný či zlý, zároveň je ale rád, že ho má, protože kdyby žil sám, všechny peníze by utratil ve městě
            \item[Lennie Small,] trošičku mentálně zaostalý, bez pomoci George by se o sebe nejspíš nepostaral, dětinský, velice silný (dokáže zabíjet zvířata tím, že je \uv{hladí}, podobně pak zabil Curleyovu ženu), nikdy ale nikoho zabít nechtěl, nedokáže ovládat svoji sílu, Small může značit jeho psychiku (jako malého dítěte)
            \item[Curley,] syn statkáře, namyšlený mladý hošík, vytahuje se, i když nemá proč, rád vyvolává konflikty, podezřívavý vůči jeho ženě, závistivý
            \item[Curleyova žena,] je doma pořád sama, navíc od ní Curley pořád odhání ostatní muže, nemá Curleyho ráda, sní o kočovném životě s divadlem, na konci zabita Lenniem, s Lenniem a Gerogem má společné zmařené sny
            \item[Candy,] starý uklízeč, nemá ruku, k úrazu přišel na farmě, takže mu dali velkou peněžní odměnu, měl velmi starého psa, který strašně smrděl, takže ho utratil
            \item[Slim,] pohledný schopný kočí, přemýšlivý, vše dělá s nadhledem
            \item[Crooks,] černoch, protože žil sám v nějaké místnosti, četl knihy, příklad rasové diskriminace, čistotný
        \end{description}
    \item[názor:] Konec byl možná trochu krutý, ale dobře vystihuje pointu díla a jakýsi \uv{uzavřený kruh smrtí}. Knihu jsem přečetl rychle, což taky značí o mém názoru na ni. 
    \item[kontext:] VeDU si POcTivÉ záPIsKy :)
    \item[zdroje:] $ $
    \begin{itemize}
        \setlength\itemsep{0em}
        \item[$-$] STEINBECK, John. \textit{o myších a lidech} [online]. Gymnázium Josefa Bořka, Česká Těšín, [cit. 2024-04-07]. Dostupné z: https://www.gmct.cz/media/files/library/PDF/O%20my%C5%A1%C3%ADch%20a%20lidech%20-%20John%20Steinbeck.pdf
    \end{itemize}
\end{description}
\end{document}
