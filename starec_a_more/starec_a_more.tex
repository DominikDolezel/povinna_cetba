\documentclass{article}
\usepackage{fullpage}
\usepackage[czech]{babel}
\usepackage{amsfonts}
\usepackage{hyperref}

\title{\vspace{-2cm}Stařec a moře (Ernest Hemingway)\vspace{-2cm}}
\date{}
\author{}

\begin{document}
\maketitle
\section{První část.}
\begin{description}
    \setlength\itemsep{0.15em}
    \item[druh:] epická próza (je jemný náznak děje, ale většinově jen popisuje pocity a myšlenky starce)
    \item[žánr:] novela (jasná jedna dějová linka s pointou -- přestože prohrál, protože dovezl jen kostru ryby, zemřela nadarmo, ale na druhé straně je vítězem v tom, že to vůbec zvládl a vybojoval si respekt ostatních rybářů, i když nezískal to, co chtěl, získal respekt ostatních a taky potvrzení, že je ještě k něčemu), může být vnímáno i jako povídka, pokud bychom tam neviděli pointu
    \item[téma:] hlavní dějová linka je o lovění velké ryby, životu na moře, ale podle mě to je jakási metafora pro život, možná z toho plyne poučení, že se člověk nemá pouštět do něčeho, na co už nemá, jasně se uplatňuje Hemingwayova metoda ledovce
    \item[motivy:] ryba, člun, kudla, létací ryby, žraloci, západ, pasát, krev, baseball, sny o lvech (zdálo se mu o lvech, o tom, jak když byl mladší, jezdil lovit k Africe)
    \item[zařazení výňatku do kontextu díla:] stařec už ulovil rybu, má ji uvázanou ke člunu a pluje zpět na pevninu, jenže ho několikrát přepadnou žraloci, kteří rybu postupně celou sní
    \item[časoprostor:] Havana či její okolí (Kuba), v první polovině dvacátých let devatenáctého století
    \item[kompoziční výstavba:] neděleno na kapitoly, kompozice chronologická s retrospektivními odbočkami (třeba silový souboj mezi \uv{starcem} a nějakým zápasníkem -- měli spolu páku a stařec vyhrál), gradace (na zač. spoustu popisných pasáží apod., ke konci už plyne velmi rychle)
\end{description}

\section{Druhá část.}
\begin{description}
    \setlength\itemsep{0.15em}
    \item[vypravěč:] nad textem, nezaujatý, často nahlíží do myšlenek postav, nejvíce tedy děj popisuje ze starcova pohletedy děj popisuje ze starcova pohledu, čtenář má velmi často pocit, že vypravěčem je stařec, některé myšlenky mu běží hlavou a některé říká nahlas
    \item[vyprávěcí způsoby:] \textit{--er} forma, v dialozích \textit{--ich}
    \item[typy promluv:] přímá, nepřímá, nevlastní přímá řeč
    \item[jazyková stránka:] spisovný jazyk, množství přechodníků či zastaralého použití \textit{s} místo \textit{z} v druhém pádu, slovní zásoba související s rybařinou, objevují se španělská slova
    \item[postavy:]
        \begin{description}
            \setlength\itemsep{0.15em}
            \item[stařec,] vlastním jménem Santigo, hlavní postava, je starý, ale velmi zkušený, silný, ale v poslední době nemá štěstí na úlovky, až se jednou vydá na moře, chytí takovou rybu, jakou nikdy předtím nikdo neviděl, ale sní mu ji žraloci na cestě ke břehu, vyčítá si, že neměl plout tak daleko od břehu, má rád chlapce, bere ho jako vlastního syna a parťáka, je silný, uvědomělý, moudrý, nevadí mu fysická bolest (např. v rukou apod.), dříve cestoval po světě, měl ženu, podrobnosti nevíme, respektuje přírodu, ale i změny, které přichází s dobou
            \item[chlapec,] Maolin, už odmala jezdil lovit se starcem, ale protože se starci nedaří a jeho rodiče jsou chudí, musí chlapec jezdit s jinými čluny (chtějí to rodiče), on se však dál stará o starce, je mu velice zavázán, záleží mu na něm, je velmi ochotný a hodný
        \end{description}
    \item[názor:] Sám asi nedokážu domyslet pravý význam této knihy. Je mi jasné, že tím autor chtěl \textit{něco} říci, ale nedokážu to rozklíčovat do hmatatelné podoby. Kniha je sice zdlouhavá, ale na konci to stojí za tu možnost přemýšlet o pointě.
    \item[kontext:]  Vedu si poctivé zápisky.
    \item[zdroje:] $ $
    \begin{itemize}
        \setlength\itemsep{0em}
        \item[$-$] HEMINGWAY, Ernest. \textit{Stařec a moře}. Online. Neuvedeno. Praha: Československý spisovatel, 1957. Dostupné z: https://www.gmct.cz/media/files/library/PDF/Sta%C5%99ec%20a%20mo%C5%99e%20-%20Ernest%20Hemingway.pdf. [cit. 2024-06-12].
    \end{itemize}
\end{description}
\end{document}
