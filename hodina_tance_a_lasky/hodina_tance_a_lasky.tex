\documentclass{article}
\usepackage{fullpage}
\usepackage[czech]{babel}
\usepackage{amsfonts}
\usepackage{hyperref}

\title{\vspace{-2cm}Hodina tance a lásky (Pavel Kohout)\vspace{-2cm}}
\date{}
\author{}

\begin{document}
\maketitle
\section{První část.}
\begin{description}
    \setlength\itemsep{0.15em}
    \item[druh:] epika (jasný děj, lyrické pasáže)
    \item[žánr:] román (rozvinutý děj, postavy se vyvíjejí, je jich velké množství)
    \item[téma:] co s člověkem udělá moc, přesvědčení, výchova; jak se stane, že skvělá idea se otočí v totalitu a stane se nečím špatným?; láska,
    \item[motivy:] SS, vlaky, foukací harmonika, děvky, válka, vojna, králíci, nůž, závist, narozeniny, bazén, balet
    \item[zařazení výňatku do kontextu díla:] ze začátku díla, kdy Kristina přijíždí na pevnost z Berlína
    \item[časoprostor:] Terezínská pevnost, kolem vylodění v Normandii, děj trvá 24 hodin, poslední kapitola je z roku 1966, Panský dům, kanceláře, bazén
    \item[kompoziční výstavba:] 21 kapitol, chronologická s výraznými retrospektivami, některé kapitoly jsou souběžně paralelní kompozice
\end{description}
\section{Druhá část.}
\begin{description}
    \setlength\itemsep{0.15em}
    \item[vypravěč:] nad dějem, objektivní, vidí postavám do hlavy, v poslední kapitole přechází do \textit{--ich} formy
    \item[vyprávěcí způsoby:] \textit{--er} forma
    \item[typy promluv:] nevlastní přímá řeč, nepřímá řeč
    \item[jazyková stránka:] hodně metaforické, spisovná čeština, časté germanismy
    \item[postavy:]
        \begin{description}
            \setlength\itemsep{0.15em}
            \item[Karel Kleinburger,] vedoucí pevnosti, prostě věřil myšlence německé nadřazenosti, nemyslel si, že páchal něco zlého, miluje svoji dceru a ženu, chce i s židy (lidmi, které považuje za méněcenné) zacházet podle zákona
            \item[Kristina,] jeho dcera, je jí právě osmnáct, je humanisticky zaměřená, nechápe moc, třeba když se zastala své učitelky baletu, byla už vychovávána v nacistickém režimu, proto to považuje za normální, pak se skrze manželství dostane do Ameriky, ale nevěří, že existovaly koncentrační tábory, myslí si, že na její rodině byla udělána křivda
            \item[Gertruda,] jeho manželka
            \item[Grube,] oportunista, chtěl si jen zachránit život a vyšplhal se až na zástupce velitele pevnosti
            \item[Monika Grubeová,] jeho maželka, mají spolu takový vztah, že ona mu zabíhá, jsou vlastně spolu jen pragmaticky, naoko nacistka, ale ve skutečnosti tomu nevěří, stejně jako její manžel
            \item[Kolatschek,] absolutně využívá situace a dělá si tam z toho děvín
            \item[Weismüller,] oddaný straně a jejím myšlenkám, taky protože tomu věřil, stejně jako Kleinburger, taky byl vychováván Stranou
            \item[Anna,] tanečnice, na vše už byla zvyklá, tak otupělá, že ztratila vlastně svoje lidství, ani se už nebrání těm nepřiměřeným rozkazům
        \end{description}
    \item[názor:]
    \item[kontext:]  \vspace{-0.5em}
        \setlength\itemsep{0em}
        \begin{itemize}
            \item[$-$]
        \end{itemize}
    \item[zdroje:] $ $
    \begin{itemize}
        \setlength\itemsep{0em}
        \item[$-$]
    \end{itemize}
\end{description}
\end{document}
