\documentclass{article}
\usepackage{fullpage}
\usepackage[czech]{babel}
\usepackage{amsfonts}
\usepackage{hyperref}

\title{\vspace{-2cm}Maryša (Alois a Vilém Mrštíkové)\vspace{-2cm}}
\date{}
\author{}

\begin{document}
\maketitle
\section{První část.}
\begin{description}
    \setlength\itemsep{0.15em}
    \item[druh:] drama
    \item[žánr:] tragédie (vražda na konci, nesplněná láska)
    \item[téma:] dopady domluvených sňatků na život člověka
    \item[motivy:] jed, káva, sekera, pivo, muzika, písně, hospoda, malovaná truhla
    \item[zařazení výňatku do kontextu díla:] Maryša otráví Vávru, svého manžela
    \item[časoprostor:] vesnice na jižní Moravě (inspiracev Těšanech), různá nářečí: naznačují, že děj se může odehrávat kdekoliv, první dvě jednání říjen 1883, zbylý tři o dva roky později (po tom, co se Francek vrátí z vojny)
    \item[kompoziční výstavba:]
\end{description}
\section{Druhá část.}
\begin{description}
    \setlength\itemsep{0.15em}
    \item[vypravěč:] není
    \item[vyprávěcí způsoby:] \textit{--ich} forma, dialogy (monology)
    \item[typy promluv:] přímá řeč
    \item[jazyková stránka:] směs moravských nářečí a brněnského hantecu
    \item[veršová výstavba:] --
    \item[postavy:]
        \begin{description}
            \setlength\itemsep{0.15em}
            \item[Maryša = Vávrová,] vdala se proti své vůli, říkala rodičům, že ho nechce, že s ním nebude spokojena, oni ji však stejně donutili, protože ji nechtěli dát chudému Franckovia Vávra měl statek, i když se říkalo, že je to špatný hospodář
            \item[Vávra,] manžel Maryšin, mlynář, násilník (svoji předchozí ženu umučil), Maryšu bije též, bouřlivý, lakomý, chcě Maryšino věno
            \item[Lízal,] otec Maryšin, sedlák, zpočátku chce Maryšu provdat za Vávru, nicméně retrospektivně si uvědomuje, že udělali chybu, když vidí, jak Maryša náhle zestárla a jak je nešťastná
            \item[Lízalka,] matka Maryšina, zlobí se na ni, protože ji neposlouchá, myslí si, že si Maryša na manžela zvykne, protože ona to tak měla taky; narozdíl od Lízala na ní náznaky lítosti po Maryšině sňatku nejsou pozorovány
            \item[Francek,] nápadník Maryšin, chudý, chtěla si ho vzít, ale rodiče nesouhlasili a navíc musel odjet na vojnu, trochu sebestředný (řekne, že bude za Maryšou chodit, ale nedomyslí, jaké to pro ni bude mít následky, že ji Vávra bude bít), matka Horačka
        \end{description}
    \item[názor:] Odlehčené pohodové čtení. Kniha se mi líbila. Zpočátku byl problém porozumět podivnému nářečí, nicméně po pár stránkách si člověk přivykne.
    \item[zdroje:] $ $
    \begin{itemize}
        \setlength\itemsep{0em}
        \item[$-$] MRŠTÍK, Alois; MRŠTÍK, Vilém. \textit{Maryša : drama v pěti jednáních}
[online]. V MKP 1. vyd. Praha : Městská knihovna v Praze, 2011
[aktuální datum citace e-knihy – př. cit. 2023-10-15]. Dostupné z
WWW: http://web2.mlp.cz/koweb/00/03/37/00/62/marysa.pdf.
    \end{itemize}
\end{description}
\end{document}
