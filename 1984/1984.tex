\documentclass{article}
\usepackage{fullpage}
\usepackage[czech]{babel}
\usepackage{amsfonts}
\usepackage{hyperref}

\title{\vspace{-2cm}1984 (George Orwell)\vspace{-2cm}}
\date{}
\author{}

\begin{document}
\maketitle
\section{První část.}
\begin{description}
    \setlength\itemsep{0.15em}
    \item[druh:] epika (jasný nosný daný děj)
    \item[žánr:] antiutopistický román, popisuje svět v totalitní společnosti, útlak lidských práv
    \item[téma:] nesvoboda v totalitární společnosti a jak jí docílit
    \item[motivy:] cigarety, telestěna, Starna, gin vítězství, mikrofony, ideopolicie, 2 plus 2, doublethink
    \item[zařazení výňatku do kontextu díla:] ze samého začátku díla, jsme uvedeni do tohoto světa a seznámeni s místními pořádky
    \item[časoprostor:] rok 1984 ve státu Oceánie, v Londýně
    \item[kompoziční výstavba:] chronologická s retrospektivními odbočkami, třeba vzpomínky na dobu před Revolucí
        síni, začátek \textit{in medias res}
\end{description}
\section{Druhá část.}
\begin{description}
    \setlength\itemsep{0.15em}
    \item[vypravěč:] vypravěč není součástí děje, ale některé věci subjektivně komentuje; někdy vypráví Winston
    \item[vyprávěcí způsoby:] \textit{--er forma}
    \item[typy promluv:] nepřímá řeč
    \item[jazyková stránka:] dílo jako takové je psáno soudobou češtinou, ale v Oceánii je vymýšlen nový jazyk (\textit{newspeak}),
        který spočívá v co možná největším osekání významů slov tak, aby tímto jazykem nebylo možné vyjádřit jakýkoliv protistátní názor
    \item[postavy:]
        \begin{description}
            \setlength\itemsep{0.15em}
           	\item[Winston,] hlavní postava, přestože je ve Vnější Straně a pracuje ve státní instituci, pochybuje o
                státním zřízení, uvažuje nad tím, jaké to bylo před Revolucí, snaží se myslet kriticky a ne jen
                věřit tomu, co říká propaganda; uvědomuje si, že nežije svobodně a že není šťastný, chce s tím něco dělat,
                je smířený s tím, že zemře
           	\item[Julie,] jeho přítelkyně, je taky proti Straně, ale snaží se chovat taky, aby to vypadalo, že je velmi loajální,
                ale svůj nesouhlas vyjadřuje soukromě, nedává jej najevo
            \item[O`Brien,] příslušník Vnitřní strany, Winston si myslí, že je taky pro Straně, ale vše to byla jen past na to,
                aby je nalákal k sobě domů a mohl je chytit; ideologii evidentně zaryle věří, přestože je velmi inteligentní
        \end{description}
    \item[názor:] Po dlouhých letech kniha, kterou jsem nečetl jen proto, že jsem ji musel číst.
    \item[kontext:]  Dělám si poctivé zápisky.
    \item[zdroje:] $ $
    \begin{itemize}
        \setlength\itemsep{0em}
        \item[$-$] ORWELL, George. \textit{1984} Online. ? ?, ? Dostupné z: \url{https://www.gmct.cz/media/files/library/PDF/1984%20-%20George%20Orwell.pdf}. [cit. 2025-02-23].
    \end{itemize}
\end{description}
\end{document}
