\documentclass{article}
\usepackage{fullpage}
\usepackage[czech]{babel}
\usepackage{amsfonts}
\usepackage{hyperref}

\title{\vspace{-2cm}RUR (Karel Čapek)\vspace{-2cm}}
\date{}
\author{}

\begin{document}
\maketitle
\section{První část.}
\begin{description}
    \setlength\itemsep{0.15em}
    \item[druh:] drama (pochmurný děj, konec lze vnímat i pozitivně i negativně -- sice vymřeli lidé, ale třeba
        se zrodilo něco lepšího, třeba roboti dostali lidské vlastnosti)
    \item[žánr:] kolektivní drama (obavy o osud lidstva)
    \item[téma:] technologický pokrok; co dělá člověka člověkem
    \item[motivy:] robot, továrna, jídlo, loď, tajný recept, elektrárna, ostrov, plot, zkumavky, výzkum
    \item[zařazení výňatku do kontextu díla:] ze začátku díla, kdy se paní Helena poprvé vidí s Dominem, přijela za ním,
        aby zastavila výrobu robotů
    \item[časoprostor:] čas ani prostor blíže neurčen, ale pravděpodobně někdy v budoucnosti na nějakém ostrově, kde je továrna;
        jinak pracovna ředitele, dům, přístav
    \item[kompoziční výstavba:] chronologická s retrospektivními odbočkami; vstupní komedie a tři dějství
\end{description}
\section{Druhá část.}
\begin{description}
    \setlength\itemsep{0.15em}
    \item[vypravěč:] není, max. scénické poznámky, jež jsou obzvláště na začátku docela podrobné, poté detaily upadají
    \item[vyprávěcí způsoby:] \textit{--ich forma}
    \item[typy promluv:] přímá řeč
    \item[jazyková stránka:] v podstatě soudobá čeština, slovní zásoba: robot
    \item[postavy:]
        \begin{description}
            \setlength\itemsep{0.15em}
           	\item[Domin,] ředitel Rossumových závodů, v roboty věří, myslí si, že jejich výroba je správná věc, chce, aby
            lidi nemuseli pracovat a aby se vymýtila bída -- socialistické myšlenky
           	\item[Fabry, Gall, Hallmeier, Busman] ředitelé jednotlivých závodů, všichni se zamilovali do
            Heleny, prootože jsou tam celou dobu sami; všichni chytří, pracovití pánové, ale možná je pro
            ně výdělek důležitější, neřeší následky výroby robotů
           	\item[Helena,] dcera prezidenta, přijela do Rossumových závodů, aby zastavila výrobu robotů, protože
            jí vadily etické otázky; nakonec se vezme s Dominem, ale nemohou spolu mít děti; na konci spálí recept
            na nějaké oživující sérum -- je prozíravá, myslí na bezpečnost lidstva, bohužel to bylo pozdě
           	\item[stavitel Alquist,] taky ředitel jednoho ze závodů, ale narozdíl od ostatních potřebuje k životu
            manuální práce -- roboti se mu moc nezamlouvají, na konci jej jako jediného nezabijí, protože pracoval rukama jako oni
        \end{description}
    \item[názor:] Knihu jsem přečetl na jeden zátah. Myslím si, že v dnešní době je s
    nástupem umělé inteligence aktuálnější než kdy dříve.
    \item[kontext:]  Dělám si poctivé zápisky.
    \item[zdroje:] $ $
    \begin{itemize}
        \setlength\itemsep{0em}
        \item[$-$] ČAPEK, Karel. \textit{RUR.} Online. ? ?, ? Dostupné z: \url{https://web2.mlp.cz/koweb/00/03/34/75/81/rur.pdf}. [cit. 2024-12-3].
    \end{itemize}
\end{description}
\end{document}
