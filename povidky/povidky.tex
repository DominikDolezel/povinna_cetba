\documentclass{article}
\usepackage{fullpage}
\usepackage[czech]{babel}
\usepackage{amsfonts}
\usepackage{hyperref}

\title{\vspace{-2cm}Povídky (Edgar Allan Poe)\vspace{-2cm}}
\date{}
\author{}

\begin{document}
\maketitle
\section{První část.}
\begin{description}
    \setlength\itemsep{0.15em}
    \item[druh:] epika (obsahuje děj)
    \item[žánr:] povídky (prozaické)
    \begin{center}
      \begin{tabular}{ l || c | l }
       & žánr & téma \\\hline
       Jáma a kyvadlo & hororová povídka & vliv alkoholu na chování člověka (nejen vůči zvířatům) \\
       Vraždy v ulici Morgue & detektivní povídka & řešení záhady pomocí analytického myšlení \\
       Maska červené smrti & hororová povídka & jistý osud při morové epidemii: před morem se nejde schovat \\
       Černý kocour & hororová povídka & vliv alkoholu na chování člověka nejen vůči zvířatům \\
       Zlatý brouk & romaneto \footnotemark & honba za pokladem, šílenství
      \end{tabular}
    \end{center}
    \footnotetext{fantasy, záhada + logické vysvětlení}
    \item[$-$] podrobněji rozebíráme Černého kocoura.
    \item[zařazení výňatku do kontextu díla:] Vypravěč měl v mládí rád zvířata. V dospělosti ovšem propadá alkoholu. Svému černému kocourovi vypíchne oko a pak ho pověsí. Natož mu shoří dům a na ohořelé stěne je k vidění podobizna oběšené kočky. Později v hopodě potká podobnou černou kočku, kterou si vezme k sobě domů. V nitru duše se jí však bojí. To tak, že chce kocoura zabít sekyrou, trefí se ovšem do své manželky. Rozhodne se, že ji zazdí, aby ji nemohl nikdo najít. Když policie poněkolikáté prohledává dům, kocour zamňouká a zazděná žena je nalezena.
    \item[časoprostor:] dům
    \item[kompoziční výstavba:] retrospektivní (vypráví, co se stalo)
\end{description}
\section{Druhá část.}
\begin{description}
    \setlength\itemsep{0.15em}
    \item[vypravěč:] hlavní postava
    \item[vyprávěcí způsoby:] \textit{--ich} forma, (vše kromě Masky červené smrti)
    \item[typy promluv:] vypravování
    \item[postavy:]
        \begin{description}
            \setlength\itemsep{0.15em}
            \item[vypravěč,] hlavní postava, v dětství milovník zvířat, propadl alkoholu, hrubý, surový
            \item[manželka vypravěčova,] milující i přes jeho závislost, později zavražděna
            \item[kocouři,] přítulní, pořád za pánem chodí, ale ten se jich chce zbavit, později se jich vyloženě bojí
        \end{description}
    \item[názor:] Poeovy povídky se mi líbily. Mám rád hororovější příběhy a toto je dobrý balanc -- jsou strašidelné tak akorát. Příběhy mě úplně vtáhly, přečetl jsem je velice rychle a bavilo mě to. Jediné, co mi vadí je, že hrdina v povídce Jáma a kyvadlo byl na konci zachráněn.
    \item[zdroje:] $ $
    \begin{itemize}
        \setlength\itemsep{0em}
        \item[$-$] POE, Edgar Allan. Jáma a kyvadlo a jiné povídky. 2. vyd. v Odeonu. Praha: Odeon, 1978. Klub čtenářů (Odeon).
    \end{itemize}
\end{description}
\end{document}
