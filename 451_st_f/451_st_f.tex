\documentclass{article}
\usepackage{fullpage}
\usepackage[czech]{babel}
\usepackage{amsfonts}
\usepackage{hyperref}

\title{\vspace{-2cm}451 stupňů Fahrenheita (Ray Bradbury)\vspace{-2cm}}
\date{}
\author{}

\begin{document}
\maketitle
\section{První část.}
\begin{description}
    \setlength\itemsep{0.15em}
    \item[druh:] epika (děj, ale časté rozsáhlé popisy), jsou tam prvky lyriky
    \item[žánr:] dystopický román, nějaké prvky sci-fi
    \item[téma:] ohloupění a zovečkovatění lidské populace (lidé si jsou cizí, baví se vlastně o ničem, nemají se rádi, jsou nešťastní, veškerá vědomost v podobě knih je pálena a snaha vyvracet nesmyslnost situace sankcionována), přehnaná korektnost ke všemu (někomu vadí, co se píše v knihách, soudí se a soud vyhraje, tak je radši zakážeme)
    \item[motivy:] Ohař, oheň, kniha, rychlá auta, široká asfaltová silnice, televizní stěna, reproduktor v uchu
    \item[zařazení výňatku do kontextu díla:] z prostřední části díla, kdy Montag ukradl knihu, má ji pod polštářem, přečetl si kousek, začne o tom přemýšlet, udělá se mu nějak špatně a nedojde do práce
    \item[časoprostor:] pravděpodobně Spolené státy americké, čas blíže neurčen, pravděpodobně někdy v budoucnosti, děj během několika dnů až málo týdnů; Montagův dům, Faberův dům, požárnická stanice, ulice, pak i venku z města, dá se odhadovat asi 25. století
    \item[kompoziční výstavba:] chronologická s retrospektivními odbočkami, děj se odehrává během několika dní, maximálně týdnů, neplyne stejně rychle -- někdy více zhuštěno a někdy více popisné, členěno na tři oddíly (Ohniště a Salamandr -- do okamžiku, když jeho žena odhalí, že má ty knihy, Síto a písek -- do momentu, než přijedou před jeho dům, Oheň se rozhoří)
\end{description}
\section{Druhá část.}
\begin{description}
    \setlength\itemsep{0.15em}
    \item[vypravěč:] vypravěč je nad dějem, často rozsáhlé popisné pasáže z pohledu Montaga, občas mu taky vidíme do hlavy
    \item[vyprávěcí způsoby:] \textit{--er forma}
    \item[typy promluv:] nepřímá řeč, přímá řeč, nevlastní přímá řeč
    \item[jazyková stránka:] spisovná čeština, žádná speciální slovní zásoba, jen v popisných pasážích jsou často metafory, přirovnání nebo barvivější slova
    \item[postavy:]
        \begin{description}
            \setlength\itemsep{0.15em}
      		\item[Guy Montag,] požárník, zajímá ho, co se píše v knihách, uvědomuje si, že lidé nejsou šťastní, snaží se na rozdíl od ostatních o věcech přemýšlet a snaží se dělat něco proto, aby se status quo změnilo, výbušný, v některých situacích nedokázal udržet klid
       	  \item[Faber,] bývalý vysokoškolský profesor, je daleko bojácnější než Montag, ale taky nesouhlasí s všeobecnou hloupostí a chce proti ní bojovat
         	\item[Mildred,] Montagova žena, představuje typickou postavu tehdejší společnosti, která uvěřila propagandě, nad věcmi nepřemýšlí a maslí si, že je šťastná
         	\item[Clarissa McClellanová,] mladá dívka, podivínka, kterou Montag potkal a se kterou si rozuměl, protože se ho ptala na zajímavé otázky, nebyla jako ostatní lidé, uvěžovala nad věcmi, používala svoje smysly, užívala si přírody, nenechávala se oblbnout propagandou, nechodí do školy
         	\item[Beatty,] velitel požárnické stanice, velmi chytrý, intelignetní, sečtělý, ale asi si uvědomuje, že je nešťastný, protože na konci pravděpodobně chce zemřít
        \end{description}
    \item[názor:] Pro mě docela zdlouhavá, ale myslím si, že vyústění a pointa za to stálo.
    \item[kontext:]  Dělám si poctivé zápisky.
    \item[zdroje:] $ $
    \begin{itemize}
        \setlength\itemsep{0em}
        \item[$-$] BRADBURY, Ray. \textit{451 stupňů Fahrenheita} Online. ? ?, ? Dostupné z: \url{https://www.milujemecestinu.cz/citanka/Bradbury-Ray---451-stupňů-Fahrenheita.pdf}. [cit. 2025-03-2].
    \end{itemize}
\end{description}
\end{document}
