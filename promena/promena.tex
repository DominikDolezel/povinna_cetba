\documentclass{article}
\usepackage{fullpage}
\usepackage[czech]{babel}
\usepackage{amsfonts}
\usepackage{hyperref}

\title{\vspace{-2cm}Proměna (Franz Kafka)\vspace{-2cm}}
\date{}
\author{}

\begin{document}
\maketitle
\section{První část.}
\begin{description}
    \setlength\itemsep{0.15em}
    \item[druh:] epika
    \item[žánr:] absurdní povídka (krátký děj, jedna hlavní dějová linka, jednoduchost, jen malé množství postav), lze vnímat i jako novelu, lze chápat i symbolicky (proměna může symbolizovat úraz apod.), místy připomíná naturalismus (formou)
    \item[téma:] Nefungující rodina, odcizení, nulová komunikace, nezájem o domluvu.
    \item[motivy:] brouk, nožičky, postel, stěny, lezení, služebná, potrava, gauč, plachta, jablko, housle, podnájemníci
    \item[zařazení výňatku do kontextu díla:] je to samý začátek díla, kdy Řehoř zjišťuje, že se proměnil v brouka
    \item[časoprostor:] nespecifikovaný čas, byt jedné rodiny v nějakém městě
    \item[kompoziční výstavba:] chronologická, tři části, retrospektivní odbočky spíše nejsou, začátek in medias res
\end{description}
\section{Druhá část.}
\begin{description}
    \setlength\itemsep{0.15em}
    \item[vypravěč:] je nad příběhem, není jeho součástí, občas se ale volně propojuje s myšlenkami Řehoře -- nevlastní přímá řeč, s ostatními postavami spíš ne
    \item[vyprávěcí způsoby:] \textit{--er forma}
    \item[typy promluv:] velké množství již dříve zmíněných nepřímých řečí, přímé řeči
    \item[jazyková stránka:] v podstatě soudobá čeština, ale vyskytují se některé historismy související s uplynulou dobou (pancíř, obchodní cestující, direktor), některé zajímavé / ne tak běžné výrazy (kožešinový boa, spadnout naznak), docela přímý popis, čímž je podporovaná pochmurná atmosféra, tak nějak nezajímavě to plyne
    \item[postavy:]
        \begin{description}
            \setlength\itemsep{0.15em}
            \item[Řehoř Samsa,] hlavní postava, jednoho dne se promění v brouka a neví, co má dělat. Je to pracovitý obchodní cestující, který svým poměrně vysokým příjmem živí celou svou rodinu, která ho tím pádem tak tročku zneužívá, protože si zvykli na ten komfort. Práce ho trápí, je náročná, ale on to bere jako danou věc, protože jeho rodiče musí splatit dluh. Před proměnou byl pracovitý a ohleduplný. Rezignuje na sůj osobní život, což je špatně. Je hodný, myslí na svoji sestru, které chce dopřát nákladné studium na konzervatoři, protože o tom touží a rodiče nemají dost peněz. Po tom, co se promění v brouka, se o něj stará v podstatě jen sestra, protože ostatní se ho štítí. Otec dokonce už od začátku nevěřil, že je to on. V některých chvílích jej chce zabít (háže po něm jablka). Umírá vyhladovělý a všichni jsou v podstatě rádi, že se tak stalo, protože se o něj aspoň nemusí starat, mohou se přestěhovat do menšího bytu bez divných pohledů sousedů (co to tam nesou?) a tak podobně. Selhání komunikace, odcizení, nefungující rodina.
            \item[Markéta,] Řehořova sestra, hraje na housle, jako jediná se o Řehoře stará, i když se ho štítí, ale po nějaké době si víc zvykne. Přemýšlí, jak by Řehořovi zpříjemnila život (posouvání nábytku, když si všimla, že leze po zdech atd.), zezačátku se mu snaží i dávat několik druhů jídla, ale později to už upadá, spíš se ho snaží odbýt (o když to nedělá záměrně, má své práce dost). Zezačátku se o něj nejvíce starala, ale na konci řekla ona, že se \uv{toho} musí zbavit.
            \item[Řehořův otec] je už starý, rodina se naučila spoléhat na Řehoře, po proměně začal pracovat, prozíravý (odkládal si nějakou část peněz, co jim dával Řehoř), je mužem činu (házení jablek po Řehořovi). Už od začátku nevěřil, že ten brouk je Řehoř, takže byl proti němu více vysazený.
            \item[Řehořova matka] je nejspíš nemocná (i když to taky může být stářím), Řehořovi chtěla taky pomáhat (dávat jídlo atd.), ale Markéta řekla, že to bude dělat sama, později, jak ho odbývá, se matka jednou sebere a uklidí mu v pokoji, protože je tam strašný nepořádek, vadil jí brouk, matka absolutně selhala, když měla nemohoucí dítě, o které se potřebovala postarat, to ale nedělala a upřednostňovala svoje pohodlí
            \item[nájemníci]
            \item[posluhovačka]
        \end{description}
    \item[názor:] Je to zajímavá povídka s psychologickým přesahem. Mám rád ten dopad / to poslání, které chce sdělit. Kniha se mi četla dobře.
    \item[kontext:]  Dělám si poctivé zápisky.
    \item[zdroje:] $ $
    \begin{itemize}
        \setlength\itemsep{0em}
        \item[$-$] KAFKA, Franz. \textit{Proměna.} Online. ? ?, ? Dostupné z: \url{https://gyza.cz/storage/brozova/franz-kafka---promstna_28.pdf}. [cit. 2024-09-14].
    \end{itemize}
\end{description}
\end{document}
