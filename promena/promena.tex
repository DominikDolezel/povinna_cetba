\documentclass{article}
\usepackage{fullpage}
\usepackage[czech]{babel}
\usepackage{amsfonts}
\usepackage{hyperref}

\title{\vspace{-2cm}Proměna (Franz Kafka)\vspace{-2cm}}
\date{}
\author{}

\begin{document}
\maketitle
\section{První část.}
\begin{description}
    \setlength\itemsep{0.15em}
    \item[druh:] epika
    \item[žánr:] absurdní povídka (krátký děj, jedna hlavní dějová linka, jednoduchost, jen malé množství postav), lze vnímat i jako novelu, lze chápat i symbolicky (proměna může symbolizovat úraz apod.), místy připomíná naturalismus (formou)
    \item[téma:] adaptace člověka a jeho okolí na novou realitu, odcizení
    \item[motivy:] brouk, nožičky, postel, stěny, lezení, služebná, potrava, gauč, plachta, jablko, housle, podnájemníci
    \item[zařazení výňatku do kontextu díla:] je to samý začátek díla, kdy Řehoř zjišťuje, že se proměnil v brouka
    \item[časoprostor:] nespecifikovaný čas, byt jedné rodiny v nějakém městě
    \item[kompoziční výstavba:] chronologická, tři části, retrospektivní odbočky spíše nejsou
\end{description}
\section{Druhá část.}
\begin{description}
    \setlength\itemsep{0.15em}
    \item[vypravěč:] je nad příběhem, není jeho součástí, občas se ale volně propojuje s myšlenkami postav -- nevlastní přímá řeč
    \item[vyprávěcí způsoby:] \textit{--er forma}
    \item[typy promluv:] velké množství již dříve zmíněných nepřímých řečí, přímé řeči
    \item[jazyková stránka:] v podstatě soudobá čeština, ale vyskytují se některé historismy související s uplynulou dobou (pancíř, obchodní cestující, direktor), některé zajímavé / ne tak běžné výrazy (kožešinový boa, spadnout naznak)
    \item[postavy:]
        \begin{description}
            \setlength\itemsep{0.15em}
            \item[Řehoř Samsa,] hlavní postava, jednoho dne se promění v brouka a neví, co má dělat. Je to pracovitý obchodní cestující, který svým poměrně vysokým příjmem živí celou svou rodinu, která si ho za to váží. Práce ho trápí, je náročná, ale on to bere jako danou věc, protože to dluží svým rodičům (to není blíže specifikováno). je hodný, myslí na svoji sestru, které chce dopřát nákladné studium na konzervatoři, protože o tom touží a rodiče nemají dost peněz. Po tom, co se promění v brouka, se o něj stará v podstatě jen sestra, protože ostatní se ho štítí. V některých chvílích jej vlastní otec chce zabít (háže po něm jablka). Umírá vyhladovělý a všichni jsou v podstatě rádi, že se tak stalo, protože se o něj aspoň nemusí start, mohou se přestěhovat do menšího bytu bez divných pohledů sousedů (co tot am nesou?) a tak podobně.
            \item[Markéta,] Řehořova sestra, hraje na housle, jako jediná se o Řehoře stará, i když se ho štítí, ale po nějaké době si víc zvykne. Přemýslí, jak by Řehořovi zpříjemnila život (posouvání nábytku, když si všimla, že leze po zdech atd.), zezačátku se mu snaží i dávat několik druhů jídla, ale později to už upadá, spíš se ho snaží odbýt a dělat si svoje vlastní věci (nové koště dobře mete).
            \item[Řehořův otec] je už starý, ale pracovitý (když se dozví, že se Řehoř proměnil, začne chodit do práce -- i když pracují i matka a Markétka), navíc prozíravý (odkládal si nějakou část peněz, co jim dával Řehoř), je mužem činu (házení jablek po Řehořovi).
            \item[Řehořova matka] je nejspíš nemocná (i když to taky může být stářím), je též pracovitá (vyšívá nějaké produkty a prodává je po tom, co se Řehoř proměnil), Řehořovi chtěla taky pomáhat (dávat jídlo atd.), ale markéta řekla, že to bude dělat sama, později, jak ho odbývá, se matka jednou sebere a uklidí mu v pokoji, protože je tam strašný nepořádek.
            \item[nájemníci]
            \item[posluhovačka]
        \end{description}
    \item[názor:] Je to zajímavá povídka s psychologickým přesahem.
    \item[kontext:]  \vspace{-0.5em}
        \setlength\itemsep{0em}
        \begin{itemize}
            \item[$-$]
        \end{itemize}
    \item[zdroje:] $ $
    \begin{itemize}
        \setlength\itemsep{0em}
        \item[$-$]
    \end{itemize}
\end{description}
\end{document}
