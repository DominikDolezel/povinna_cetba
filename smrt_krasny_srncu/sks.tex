\documentclass{article}
\usepackage{fullpage}
\usepackage[czech]{babel}
\usepackage{amsfonts}
\usepackage{hyperref}

\title{\vspace{-2cm}Smrt krásných srnců (Ota Pavel)\vspace{-2cm}}
\date{}
\author{}

\begin{document}
\maketitle
\section{První část.}
\begin{description}
    \setlength\itemsep{0.15em}
    \item[druh:] epika (ale obsahuje nějaké lyrické pasáže)
    \item[žánr:] soubor osmi povídek
    \item[téma:] život rodiny Židů ve válečném (před, během i po válce) Československu, jednotlivé povídky:
    \begin{itemize}
    \item Nejdražší ve střední Evropě: tatínek koupí za draho rybník, kde je jen jedna ryba, takže přijde o hodně peněz. Pak si ten člověk, od kterého rybník koupil, přijde k němu pro ledničku, tatínek situace využije a tak se pomstí.
   	\item Ve službách Švédska: tatínkovi se náramně daří prodávat ledničky a vysavače a líbí se mu žena ředitele podniku Irma. Uvědomuje si, že u ní nemá šanci, ale čirou náhodou se seznámí s vyhlášeným malířem Nejezchlebem, od kterého by se paní Irma tuze ráda nechala namalovat. Po dlouhém napínání a předcházení si pana Nejezchleba ze strany tatínka to však nakonec nedopadne...
   	\item Smrt krásných srnců: z předválečného období, popisuje pytlácké praktiky pana Proška a jeho psa Holana. O pár let později, za války, přijde předvolání synů do koncentračního tábora, a tak se otec snaží sehnat jim něco posledního k snědku -- pytlačí srnce.
   	\item Kapři pro wehrmacht: po tom, co tatínkovi vzali jeho Buštěhradský rybník, se chtěl morálně pomstít, ukradl z něj tedy kapry těsně před tím, co odjel do koncentračního tábora
   	\item Jak jsme se střetli s Vlky: Vlci (Vlkové?) jsou rod žijící u řeky, takže mají velkou rybářskou zkušenost a tradici. Jednou se s nimi rodinka Popperových chtěla a střetnout v lovení ryb, ale prohráli.
   	\item Otázka hmyzu vyřešena: nová mucholapka dr. Jehličky jde na dračku do té doby, než ji lidé začnou používat...
   	\item Prase nebude!: maminka s tatínkem se ujali práce v místní JZD, kde chovali prasata s vidinou zaslouženého výdělku, prasata však byla znárodněna a odvezena
   	\item Králíci s moudrýma očima: Popperovi prodali domeček a koupili si chatu u Radotína. Měli málo peněz, a tak tatínek začal chovat a prodávat králíky. S vidinou velké výhry je přihlásil na soutěž, jenže králící neměli udělanou pedikúru.
    \end{itemize}
    \item[motivy:] Electrolux, made in Sweden, buick, srnci, louka, ryby a rybaření, prodávání
    \item[zařazení výňatku do kontextu díla:] z povídky Nejdražší ve střední Evropě, ze závěru povídky těsně předtím, než se dozvíme, že rybník byl prázdný
    \item[časoprostor:] v Čechách na Křivoklátsku (Buštěhrad, Radotín, Kročehlavy, řeka Berounka), Praha, období od okupace po konec druhé světové války
    \item[kompoziční výstavba:] chronologická s retrospektivními odbočkami, mezi povídkami dlouhá doba, někdy skoky, řetězová kompozice -- povídky jsou spojeny stejnými postavami
\end{description}
\section{Druhá část.}
\begin{description}
    \setlength\itemsep{0.15em}
    \item[vypravěč:] vypravěč je syn tatínka, což je vlastně jedna z hlavních postav, věci komentuje, říká na ně svoje subjektivní názory
    \item[vyprávěcí způsoby:] \textit{--ich forma}, ale v některých příbězích vypravěč vůbec nefiguruje, vlastně vypráví o tatínkovi, takže tam je \textit{--er forma}
    \item[typy promluv:] nepřímá řeč, přímá řeč
    \item[jazyková stránka:] spisovná čeština, někdy výrazy z obecné češtiny, časté anglikanismy: Made in Sweden apod., časté zkratky a názvy firem: Electrolux, PRAGA apod., obrazná pojmenování
    \item[postavy:]
        \begin{description}
            \setlength\itemsep{0.15em}
 			\item[tatínek,] rád rybaří, je pracovitý, velice dobrý prodejce, vždy, když je pro něco zapálený, nedokáže myslet na nic jiného, velmi podnikavý, cílevědomý
            \item[maminka,] velmi klidná, pragmatická, akceptuje tatínkovu šílenost, trpělivá
        \end{description}
    \item[názor:] Oddechové a docela vtipné čtení na odreagování, přestože zachycuje složitou dobu a témata.
    \item[kontext:]  Dělám si poctivé zápisky.
    \item[zdroje:] $ $
    \begin{itemize}
        \setlength\itemsep{0em}
        \item[$-$] PAVEL, Ota. \textit{Smrt krásných srnců} Online. ? ?, ? Dostupné z: \url{https://web2.mlp.cz/koweb/00/04/52/92/27/smrt_krasnych_srncu.pdf}. [cit. 2025-03-23].
    \end{itemize}
\end{description}
\end{document}
