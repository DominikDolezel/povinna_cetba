\documentclass{article}
\usepackage{fullpage}
\usepackage[czech]{babel}
\usepackage{amsfonts}
\usepackage{hyperref}

\title{\vspace{-2cm}Bílá nemoc (Karel Čapek)\vspace{-2cm}}
\date{}
\author{}

\begin{document}
\maketitle
\section{První část.}
\begin{description}
    \setlength\itemsep{0.15em}
    \item[druh:] drama
    \item[žánr:] tragédie (pochmurný průběh, smrt jedno z hlavních témat, příchod války, tragický konec)
    \item[téma:] limity lidského poznání, lidských schopností; hrozba diktátorských režimů, co zmůže jeden člověk ve společnosti?
    \item[motivy:] bílá skvrna, nemocnice, smrt, zkumavka, válka
    \item[zařazení výňatku do kontextu díla:] ke konci díla, dr. Galén objevil lék na bílou nemoc a požaduje, aby byl uzavřen
        světový mír, odmítá léčit vlivné a bohaté lidi, aby je vydíral
    \item[časoprostor:] čas ani prostor blíže neurčen, ale pravděpodobně někdy v budoucnosti
        jinak nemocnice, byt rodiny, náměstí, pracovna
    \item[kompoziční výstavba:] chronologická s retrospektivními odbočkami; tři akty, čtrnáct obrazů, dramatický oblouk
    (na konci byla naděje, že by maršál nechal uzavřít mír, aby ho dr. Galén uzdravil, ale ten byl zabit -- katastrofa)
\end{description}
\section{Druhá část.}
\begin{description}
    \setlength\itemsep{0.15em}
    \item[vypravěč:] není
    \item[vyprávěcí způsoby:] \textit{--ich forma}
    \item[typy promluv:] přímá řeč
    \item[jazyková stránka:] v podstatě soudobá čeština
    \item[postavy:]
        \begin{description}
            \setlength\itemsep{0.15em}
           	\item[dr. Galén,] velice šikovný doktor, vynalezl lék na bílou nemoc, ale nechce ji rozšířit, chce konec válek,
                protože byl jako lékař na frontě a musel ošetřovat nevinné lidi, tvrdohlavý, léčí jen chudé, protože
                ti s tím nemůžou nic dělat, v podstatě porušuje lékařskou přísahu, používá tuto situaci k vydírání a k
                dosažení svých cílů
            \item[Maršál,] fanatický diktátor, potřebuje válčit za každou cenu, vidí v tom smysl svého života, smysl národa,
                ale musí být schopný, protože o svách názorech uvěřitelně přesvědčil veřejnost; lidské životy jsou mu lhostejné,
                až na jeho jediného kamaráda Krüga, který dostal bílou nemoc, jenže galén ho odmítá léčit; pořád nechce
                ustoupit z válk a uzavřít mír -- to slíbí, až když nemoc dostane sám (nerovnost ve společnosti)
            \item[Krüg] vydělává z války, vlastní závody na výrobu zbraní, neláme si s tím hlavu, nevidí to jako něco špatného,
                prostě to chápe jako svoji obživu, byznys, je věrný maršálovi (asi taky částečně protože díky němu má co prodávat)
           	\item[profesor Sigelius] jde mu jen o prestiž, ne o vědu, je pro něj důležité, aby objev proběhl na jeho ústavu,
                ale na druhou stranu když za ním Galén na začátku přijde, nechce mu poskytnout podmínky
            \item[rodina] náhodná rodina nám v průběhu celé hry poskytuje pohled, názory, nálady průměrného obyvatelstva
        \end{description}
    \item[názor:] hra ukazuje, že ne vše je černobílé. Sám nevím, jestli mám dr. Galéna vnímat jako pozitivní,
    nebo jako negativní postavu. Za mě hezky napsáno.
    \item[kontext:]  Dělám si poctivé zápisky.
    \item[zdroje:] $ $
    \begin{itemize}
        \setlength\itemsep{0em}
        \item[$-$] ČAPEK, Karel. \textit{Bílá nemoc.} Online. ? ?, ? Dostupné z: \url{https://web2.mlp.cz/koweb/00/03/34/75/28/bila_nemoc.pdf}. [cit. 2024-12-22].
    \end{itemize}
\end{description}
\end{document}
