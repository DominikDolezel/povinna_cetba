\documentclass{article}
\usepackage[czech]{babel}
\usepackage{fullpage}
\usepackage{amsfonts}
\usepackage{hyperref}

\title{\vspace{-2cm}Krysař (Viktor Dyk)\vspace{-2cm}}
\date{}
\author{}

\begin{document}
\maketitle
\section{První část.}
\begin{description}
    \setlength\itemsep{0.15em}
    \item[druh:] epika
    \item[žánr:] novela (kratší děj, jedna hlavní dějová linka, závěrečná pointa, málo postav)
    \item[téma:] láska mezi Krysařem a Agnes, pomsta ze nezaplaacené služby městu Hammeln
    \item[motivy:] hospoda U Žíznivého člověka, hora Koppel, píšťala, peníze, loďka, řeka, dítě, země Sedmihradská
    \item[zařazení výňatku do kontextu díla:] Výňatek je ze samého začátku díla, kdy se Agnes sblíží s krysařem.
    \item[časoprostor:] hansovní město Hammeln, doba neurčena (nejspíš středověk)
    \item[kompoziční výstavba:] exposice (příchod krysaře do města, seznámění s Agnes, opětování lásky), kolise (roztržka v hospodě -- troufalý krysař požádá konšely o mzdu, jež mu právem připadá, konšelé mu však odmítají zaplatit, vymluví se na to, že smlouva, již podepsali, nebyla platná), krise (Agnes otěhotní se svým milencem Krisitiánem, krysař odchází, Agnes páchá sebevraždu), peripetie (krysař hledá Agnes, marně), katastrofa (krysař se bezvýchodnou situaci rozhodne řešit stejně jako Agnes -- vzpomene si na zemi Sedmihradskou, do níž mají být brány v propasti u hory Koppel, krysař začne silně hrát na píšťalu -- tak silně, že všichni obyvatelé hansovního města Hammeln, až na Seppa Jörgena, kterému vše dochází pomaleji, a nemluvně, následují ho na tuto horu a s nadějí na lepší život v zemi Sedmihradské vydají se na cestu do propasti); členěno na kapitoly (26), chronologicky s retrospektivními odbočkami do krysařova života
\end{description}
\section{Druhá část.}
\begin{description}
    \setlength\itemsep{0.15em}
    \item[vypravěč:] neznámý, nad dějem, vypráví objektivně
    \item[vyprávěcí způsoby:] hlavně \textit{--er} forma, v dialozích i \textit{--ich} forma
    \item[typy promluv:] přímá řeč (a uvozovací věty), nepřímá řeč, myšlenky postav
    \item[jazyková stránka:] spisovná čeština, velké množství přechodníků, metafor a obrazných pojmenování, kratší věty pro zachycení spádu
    \item[postavy:]
        \begin{description}
            \setlength\itemsep{0.15em}
            \item[krysař,] který přijde do hansovního města Hammeln s cílem vyhnat krysy, zcestovalý, moudrý, Agnes je první žena, do které se zamiloval, po tom, co mu konšelé nechtějí vyplatit mzdu za odvedení krys se rozhodne pomstít celému městu, lidé na něj pohlížejí shora
            \item[Agnes,] milenka krysařova a Kristiána, oddaná je však krysařovi, se kterým by nejradši žila, jenže otěhotní s Kristiánem, má se za něj vdávat a to nechce -- proto se rozhodne vzít si život, nevidíme do jejích pocitů
            \item[Kristián,] snoubenec Agnes
            \item[Sepp Jörgen,] zpomalený, vše mu dochází až druhý den, ale dobrý člověk
            \item[konšelé,] pyšní, nespravedliví vůči krysařovi
        \end{description}
    \item[názor:] Kniha -- a zvlášť její závěr -- mne zaujala. Jednoduchý děj, který je však napsaný velmi zajímavým jazykem, mne vtáhl a měl jsem přečteno docela rychle.
    \item[kontext:] Vedu si poctivé zápisky v elektronické podobě.
    \item[zdroje:] $ $
    \begin{itemize}
        \setlength\itemsep{0em}
        \item[$-$] DYK, Viktor. \textit{Krysař}  [online]. Praha: Městská knihovna v Praze, 2011 [cit. 2024-01-07]. Dostupné z: https://web2.mlp.cz/koweb/00/03/37/00/37/krysar.pdf
    \end{itemize}
\end{description}
\end{document}
