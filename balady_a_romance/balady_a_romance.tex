\documentclass{article}
\usepackage{fullpage}
\usepackage[czech]{babel}
\usepackage{amsfonts}
\usepackage{hyperref}

\title{\vspace{-2cm}Balady a romance (Jan Neruda)\vspace{-2cm}}
\date{}
\author{}

\begin{document}
\maketitle
\section{První část.}
\begin{description}
    \setlength\itemsep{0.15em}
    \item[druh:] lyricko-epické
    \item[žánr:] balady (báseň s pochmurným dějem a tragickým koncem) a romance (báseň s optimistickým dějem, romantickými prvky a dobrým koncem), každá báseň má v názvu \uv{balada} nebo \uv{romance}, to však nemusí obsahu odpovídat
    \item[téma:] náboženství, biblické motivy, vlastenecké prvky, láska mezi amtkou a dítětem, vesnické slavnosti, svatby, radost ze života, historické momenty a osobnosti, vina a trest
    \item[motivy:] \begin{enumerate}
        \vspace{-0.5em}
        \setlength\itemsep{0.15em}
        \item \textbf{Balada pašijová}: Bůh pošle Ježíše na zem, Satan s tím není spokojen, Bůh tedy na Ježíše pošle utrpení -- jeho matka musí sledovat jeho ukřižování
        \item \textbf{Balada horská}: dítě vezme léčivé byliny, použije je na obraz Ježíše, kterému se na něm zacelí rány
        \item \textbf{Balada dětská}: matka dítěte usne, její dítě je nemocné, přijde Smrtka a chce s ním jít pryč, dítě vzdoruje, Smrtka si ho však nakonec odvede
        \item \textbf{Balada česká}: bohyně Vesna palečkovi slíbí jedno přání, on se chce vždy na jaře vrátit na osm dní na zem a žít, to se mu splní
        \item \textbf{Romance o černém jezeře}: popisuje dávné hrdiny českého národa, na které se zapomnělo
        \item \textbf{Romance o Karlu IV.}: ne všechno je takové, jaké se zdá na první pohled, Karel IV. a Bušek z Vilhartic popíjí víno, Karlu zpočátku nechutná, Buškovi ano, později však král změní názor, stejně jako s českým národem
        \item \textbf{Romance o jaře 1848}: revoluční rok 1848, označuje revoluci jako začátek něčeho nového, pozitivního, touha po demokratických svobodách
        \item \textbf{Romance italská}: člověk, který se odvrátí od křesťanství je popraven
        \item \textbf{Romance helgolandská}: člověk na pevnině cíleně špatně naviguje loď, protože chce věno pro svoji dceru, která se bude vdávat, nakonec zjistí, že na té lodi byl její budoucí manžel
        \item \textbf{Balada zimní}: čaroděj oživí tři lidi, ti konají hříchy, nakonec jsou všichni potrestáni
        \item \textbf{Balada stará -- stará!}: žena porodí dítě, to jí nabízí, že jí naloví ryby, vypere pleny, ta nechce, otec je zanedbává, nakonec se zabijí
        \item \textbf{Balada tříkrálová}: k Jezulátkou přijdou tři králové, zahrnují ho dary, on jim řekne, že se mu vnucují, až bude kráčet na smrt, zapřou ho
        \item \textbf{Romance štědrovečerní}: Petrovi se zdá, že oslavují Ježíška, přijdu nějaké dívky, špulí pusu na Ježíška, Marii to nevadí, Josefovi to však vadí, protože se mu líbí Andulka
        \item \textbf{Balada májová}: dívka prosí Marii o jakéhokoliv chlapce, kromě zrzavého, ta jí však řekne, že už jiný není k dostání, nakonec tedy sleví
        \item \textbf{Balada rajská}: Marie jde z ráje a potká svatou Alžbetu, která si stěžuje, že se nudí, protože nejsou nehřešící ženy a jediná, která taková byla, hned zesmilnila
        \item \textbf{Balada o duši Krala Borovského}: Borovský klepá na brány nebe, oni se ho zeptají, jestli nezná nějakou motlitbu, ten jim začne odříkávat svůj epigram Svatý Jene z Nepomuku, nakonec ho tam pustí
        \item \textbf{Balada o polce}: personifikování polky, která přijíždí do vesnice a přináší s sebou radost
        \item \textbf{Balada o svatbě v Kanaán}: svatba v Kanaánu, kde není dostatek vína, Ježíš tedy vodu promění ve víno, na konci je naznačeno, že Ježíš zemře
    \end{enumerate}
    \item[zařazení výňatku do kontextu díla:]
    \item[časoprostor:] různé, většinou venkovské prostředí, čas většinou neurčitý
    \item[kompoziční výstavba:] podle jednotlivých básní, verše spíš vázané
\end{description}
\section{Druhá část.}
\begin{description}
    \setlength\itemsep{0.15em}
    \item[vypravěč:]
    \item[vyprávěcí způsoby:]
    \item[typy promluv:]
    \item[jazyková stránka:]
    \item[veršová výstavba:]
    \item[názor:]
    \item[kontext:]  \vspace{-0.5em}
        \setlength\itemsep{0em}
        \begin{itemize}
            \item[$-$]
        \end{itemize}
    \item[zdroje:] $ $
    \begin{itemize}
        \setlength\itemsep{0em}
        \item[$-$]
    \end{itemize}
\end{description}
\end{document}
