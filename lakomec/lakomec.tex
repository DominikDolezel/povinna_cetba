\documentclass{article}
\usepackage{fullpage}
\usepackage[czech]{babel}
\usepackage{amsfonts}
\usepackage{hyperref}

\title{\vspace{-2cm}Lakomec (Molière)\vspace{-2cm}}
\date{}
\author{}

\begin{document}
\maketitle
\section{První část.}
\begin{description}
    \setlength\itemsep{0.15em}
    \item[druh:] drama
    \item[žánr:] komedie (dobrý konec, komické prvky, prozaické)
    \item[téma:] láska otce a syna k jedné ženě, lakomství
    \item[motivy:] peníze, truhla, koně, zámek, drahé oblečení, kočár, krádež, vyšetřování, lakomství, překvapení, vzpoura vůči otci, prsten; inspirováno Komedií o hrnci
    \item[zařazení výňatku do kontextu díla:] Čipera ukradl Lakomci = Harpagonovi pokladnici s jeho penězi. Ten se zděsí, následeuje vyšetřování. Nejprve se nedorozuměním svěří Valer, že se oženil s jeho dcerou, kterou si Harpagon zamění za onu pokladnici. Jakmile mu to dojde, musí říci jejímu nápadníkovi, co se děje, a tak se dozví, že je to otec jeho budoucí ženy a otec Valera.
    \item[časoprostor:] Paříž -- v Harpagonově domě, zahrada; koncem 17. století
    \item[kompoziční výstavba:] chronologická, 5 jednání o několika výstupech, Aristotelova jednota místa, času a děje platí, dramatický oblouk též, Anselm zastává funkci \textit{deux ex machina}, retrospektivně se vypráví o Anselmových osudech
\end{description}
\section{Druhá část.}
\begin{description}
    \setlength\itemsep{0.15em}
    \item[vypravěč:] není
    \item[vyprávěcí způsopby:] \textit{--ich} forma, dialogy, (monology)
    \item[typy promluv:] přímá řeč
    \item[jazyková stránka:] běžná čeština s hovorovými prvky
    \item[veršová výstavba:] není, psáno prózou
    \item[postavy:]
        \begin{description}
            \setlength\itemsep{0.15em}
            \item[Harpagon,] lakomec, starý, lakomý, posedlý penězi, protivný, neempatický
            \item[Kleant,] harpagonův syn, hodný, šlechetný, milenec Mariany
            \item[Elisa,] Kleantova sestra, hodná, bojí se o svého otce, chápavá, milenka Valera
            \item[Valer,] správce Harpagonova sídla, přistěhovalec, obětavý, vychytralý, manipulativní, chce dobře vycházet s Harpagonem
            \item[Mariana,] Valerova sestra, přistěhovalkyně, krásná, chudá, žje s matkou, dcera Anselma
            \item[Frosina,] dohodila Marianu Harpagonovi
            \item[Jakub,] kočí a kuchař
            \item[dále] sluha, policejní komisař
        \end{description}
    \item[názor:] Kniha se mi líbila, zvláště její hravý děj a nečekané vyústění. Přistihl jsem se, jak sympatizuji s Harpagonovými dětmi a je mi jich líto. To je známka kvality tohoto literárního díla.
    \item[zdroje:] $ $
    \begin{itemize}
        \setlength\itemsep{0em}
        \item[$-$] MOLIÈRE. Lakomec [online]. [cit. 2022-11-19]. \\Dostupné z: \url{https://web2.mlp.cz/koweb/00/03/66/08/78/lakomec.pdf}
    \end{itemize}
\end{description}
\end{document}
