\documentclass{article}
\usepackage{fullpage}
\usepackage[czech]{babel}
\usepackage{amsfonts}
\usepackage{hyperref}

\title{\vspace{-2cm}Král Oidipus (Sofokles)\vspace{-2cm}}
\date{}

\begin{document}
\maketitle
\section{První část.}
\begin{description}
    \setlength\itemsep{0.15em}
    \item[druh:] drama (psáno pro jeviště)
    \item[žánr:] tragédie (syn zabije otce, má za manželku vlastní matku, mor postihne město, musí ho opustit, propíchne si oči); \textit{tragédie osudu} -- nedá se mu vyhnout
    \item[téma:] hledání vraha bývalého krále Láia, nevyhnutelný osud
    \item[motivy:] věštba, pastýři, hádanky, odhození dítěte, vypíchnutí očí
    \item[zařazení výňatku do kontextu díla:] věštec Teirésiás předpovídá Oidipovi těsně po tom, co nechal hledat po vrahovi Láia, že sám Oidipus je jeho vrahem + další podrobnosti (jeho syn, atd.) a že musí odejít, jinak město postihne mor. Oidipus věštci nevěří a označí ho za hlupáka. Nakonec se však ukáže, že měl pravdu, jeho manželka a matka Iokasté se oběsí a Oidipus si propíchne oči.
    \item[časoprostor:] Antika, Théby, asi 5. st. př. n. l., náměstí před královským palácem
    \item[kompoziční výstavba:] díly ani kapitoly nejsou; Na začátku je prozrazeno, co se stalo před hrou. Samotná hra je vyprávěna chronologicky s retrospektivními odbočkami do dávné doby (Oidipovo narození, Láiovo přepadení, \dots). \textit{Aristotelova jednota místa, času a děje} je dodržena. Pozorujeme expozici, kolizi, krizi, peripetii a katastrofu.
\end{description}
\section{Druhá část.}
\begin{description}
    \setlength\itemsep{0.15em}
    \item[vypravěč:] (supluje:) \textit{chorus}, scénické poznámky
    \item[vyprávěcí způsopby:] \textit{--ich} forma, občas \textit{--er}; dialogy
    \item[typy promluv:] přímá řeč, v úvodu vyprávění
    \item[jazyková stránka:] archaický, hodně inverzí, metafory
    \item[veršová výstavba:] verš vázaný, jamb, rýmy nejsou
    \item[postavy:]
        \begin{description}
            \setlength\itemsep{0.15em}
            \item[Oidipus,] král Thébský; nechce věřit, že zabil otce a má děti se svou matkou. Jakmile se však přesvědčí o pravdě, zodpovědně ji přijme a je připraven nést následky. Je soucitný -- odešel od nevlastních rodičů kvůli věštbě, která říkala, že zabije otce a bude mít děti s matkou. Moudrý -- rozluštil hádanku obávané Sfingy. Je výbušný, prchlivý.
            \item[Iokasté,] královna, Oidipova matka a žena; Věří věštbám -- proto odhodila Oidipa, upřednostňuje zájmy ostatních před svými. Jakmile se dozví, že její syn je \uv{velký} Oidipus, chce ho chránit od pravdy, kterou sama nedokáže unést (oběsila se).
            \item[Kreón,] má Oidipa rád a chce pro něj to nejlepší
            \item[Teiresias,] věštec; nenechá si vše líbit (jak s ním Oidipus) jedná
        \end{description}
    \item[názor:] předem jsem neznal rozuzlení \dots líbilo se mi, nápaditý a nečekaný vývoj situace
    \item[kontext:] Sofokles (5. st. př. n. l.) \vspace{-0.5em}
        \setlength\itemsep{0em}
        \begin{itemize}
            \item[$-$] attické období (5. -- 4. st. př. n. l.)
            \item[$-$] starověké Řecko
            \item[$-$] tragédie
            \item[$-$] Elektra, Antigona
            \item[$-$] --prosím doplnit, viz sešit--
        \end{itemize}
    \item[zdroje:] $ $
    \begin{itemize}
        \setlength\itemsep{0em}
        \item[$-$] SOFOKLES. \textit{Král Oidipús} [online]. \\
        Dostupné z: \url{https://is.muni.cz/el/1421/podzim2010/DVHs155/Stiebitz.pdf}
    \end{itemize}
\end{description}

\end{document}
