\documentclass{article}
\usepackage{fullpage}
\usepackage[czech]{babel}
\usepackage{amsfonts}
\usepackage{hyperref}

\title{\vspace{-2cm}Edison (Vítězslav Nezval)\vspace{-2cm}}
\date{}
\author{}

\begin{document}
\maketitle
\section{První část.}
\begin{description}
    \setlength\itemsep{0.15em}
    \item[druh:] lyricko-epické (rozsáhlé popisy, drobný náznak příběhu, ale není ani ucelený, ani uzavřený, lyrika jasně převažuje), je to trochu v duchu poetismu (řetězení myšlenek), ale už je to velmi doznívající (závažnější témata, už není taková hravost, daleko promyšlenější a pestřejší)
    \item[žánr:] báseň, připomíná pásmo (chybí interpunkce, řetězení, žazení obrazů za sebe, tematické; ale témata jsou víc ucelená, je tu vázaný verš)
    \item[téma:] obdivování Edisona a vynálezů, pokroku jako takových, zastřešující téma je hledání smyslu života a nacházení ho v práci, smysluplné činnosti, kontrast mezi sebevraždou a člověkem, který život zachaňuje, pokrok, trochu v duchu futurismu
    \item[motivy:] hazard, most, Praha, stín, sebevražda, žárovka, alkohol, Edison
    \item[zařazení výňatku do kontextu díla:] začátek druhé části, autor popisuje Edisonovo mládí a jeho domovinu, v dalších slokách vzpomíná na Edisonova otce
    \item[časoprostor:] různé -- zezačátku Praha, poté Amerika, na konci se vracíme zpět do Prahy; čas asi někdy ze začátku dvacátého století, možná 20. léta, s přesahem k Edisonovi do druhé poloviny 19. st.
    \item[kompoziční výstavba:] děleno do 5 částí, ty jsou dále děleny do slok, sloky nejsou stejně dlouhé, refrén: \uv{bylo tu však něco krásného co drtí, odvaha a radost z života i smrti}, paralelismus mezi jednotlivými věcmi, které splňují tento refrén, na konci se zase vracíme do Prahy a do současnosti, rámcová kompozice (rámec: jednoduchý příběh lyrického subjektu, který jde domů, vidí tam noviny a jde spát, do něj jsou vloženy všechny ty myšlenky), v prvním zpěvu se lyrický subjekt vrací z herny, najednou vidí sebevraha (je to jeho vlastní stín), chce skočit z mostu (úzkost, stesk), neskočí a přichází domů, tam otevře noviny a přemýšlí o Edisonovia  o technickém pokroku, to mu zlepší náladu, nutí ho přemýšlet o podstatě, smyslu lidského života, z počátku velká deprese, která postupně mizí, ve druhém zpěvu líčí mládí a dětství Edisona, vzpomínky na otce Edisona a otce lyrického subjektu, je tam ještě přetím, než se z něj stal vynálezce, ve třetím zpěvu už je Edison starší, autor vychvaluje další vynálezce, tvrdá práce a spousta pochybností, obdiv k technickému pokroku, určitý futurismus, řetězení metafor o elektrickém osvětlení, ve čtvrtém zpěvu zjišťuje, že ne každý je úspěšný, a že i když je za tím práce, je za tím taky náhoda, každý talentovaný člověk má možnost se projevit, ale ne vždycky to vyjde, jen některý z talentovaných dokáže ovlivnit svět, v pátém zpěvu se vracíme k lyrickému subjektu do Prahy, jde spát, myšlenky jsou více optimistické, ale pořád si uvědomuje klady a zápory, sumář života -- vyrovnat se s tím, že jsou smutné okamžiky, ale je potřeba jít dál a snažit se, směřuje k reálnému vidění skutečnosti
\end{description}
\section{Druhá část.}
\begin{description}
    \setlength\itemsep{0.15em}
    \item[vypravěč:] je součástí děje, prochází se po Praze a potká tam stín hazardního hráče (který chce spáchat sebevraždu, je to jeho vlastní stín), pak jde domů, kde v novinách vidí článek o Edisonovi, následně o něm přemýšlí
    \item[vyprávěcí způsoby:] \textit{--ich} i \textit{--er} forma podle toho, jestli se nacházíme ve vypravěčových myšlenkách, nebo ve úvahách o Edisonovi, polopřímé řeči
    \item[typy promluv:]
    \item[jazyková stránka:] v podstatě současná čeština, ale barvité, spousta metafor
    \item[veršová výstavba:] rým sdružený, vázaný verš, proto to není pásmo
    \item[postavy:]
        \begin{description}
            \setlength\itemsep{0.15em}
            \item[vypravěč,] je součástí děje, jde po Praze a najednou vidí svůj stín, stín hazardního hráče, jakmile dojde domů a rozsvítí lampu, stín zmizí, poté oslavuje Edisona za to, že mu vynalezl žárovku a on nemusí být stínem konfrontován
            \item[hazardní hráč,] je to vlastně vypravěčův stín, chce spáchat sebevraždu, protože nemá štěstí
            \item[Edison] vynálezce, vymyslel tisíce vynálezů, ale jen jeden z nich je opravdu důležitý -- žárovka
        \end{description}
    \item[názor:] Báseň se mi líbila, i když jsem jí zpožátku moc nerozuměl. To je na druhou stranu dobře, protože o ní můžu přemýšlet.
    \item[kontext:]  Vedu si poctivé zápisky.
    \item[zdroje:] $ $
    \begin{itemize}
        \setlength\itemsep{0em}
        \item[$-$] NEZVAL, Vítězslav. \textit{Edison.} Online. Dostupné z: \url{https://www.gmct.cz/media/files/library/PDF/Edison%20-%20V%C3%ADt%C4%9Bzslav%20Nezval.pdf}. [cit. 2024-10-23].

    \end{itemize}
\end{description}
\end{document}
