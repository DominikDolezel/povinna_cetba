\documentclass{article}
\usepackage{fullpage}
\usepackage[czech]{babel}
\usepackage{amsfonts}
\usepackage{hyperref}

\title{\vspace{-2cm}Farma zvířat (George Orwell)\vspace{-2cm}}
\date{}
\author{}

\begin{document}
\maketitle
\section{První část.}
\begin{description}
    \setlength\itemsep{0.15em}
    \item[druh:] epika s lyrickými prvky (píseň, báseň)
    \item[žánr:] antiutopická novela, zároveň založeno na stejném principu jako bajka
    \item[téma:] Přestože je někdy revoluce prováděna s dobrými úmysly, vládnoucí vrstva nakonec zneužije své moci a přejde se opět do totality.
    \item[motivy:] prase, mlýn, dynamo, křída, dům, postel, čtení, přikázání, píseň Zvířata Anglie, alegorie na SSSR, důležitost vzdělání
    \item[zařazení výňatku do kontextu díla:] z druhé poloviny díla, prasata začínají spát v postelích, porušují sedm přikázání, spí v domě, manipulace zvířat
    \item[časoprostor:] na Panské farmě v Anglii, pravděpodobně ve 20. století, byla napsána za druhé světové války, dlouho nemohla vyjít
    \item[kompoziční výstavba:] chronologická s retrospektivními odbočkami (ale ne moc, protože si to zvířata nepamatují), postupná gradace, cyklické (končí tam, kde to začalo), čas se postupně zrychluje, deset kapitol
\end{description}
\section{Druhá část.}
\begin{description}
    \setlength\itemsep{0.15em}
    \item[vypravěč:] nad dějem, nevyjadřuje vlastní názor, není jedna z postav
    \item[vyprávěcí způsoby:] \textit{--er forma}
    \item[typy promluv:] nepřímá řeč, přímá řeč, nevlastní přímá i polopřímá řeč
    \item[jazyková stránka:] spisovná čeština
    \item[postavy:]
        \begin{description}
            \setlength\itemsep{0.15em}
 			      \item[Napoleon,] vůdce farmy, v podstatě ze sebe moudrým způsobem udělal diktátora, jeho jediným cílem je ovládnout ostatní zvířata a sým se mít dobře, je mu úplně jedno, jak si vedou ostatní, symbolizuje Stalina
   	        \item[Pištík,] jeho přisluhovač / mluvčí, v podstatě stejného charakteru, velmi zdatný manipulátor, symbolizuje propagandu
           	\item[Kuliš,] nepřítel Napoleona, byl vyhnán a prohlášen za zrádce farmy, později se vlastně poszměňuje minulost, kdy o něm
            říkají, že se spolčil s Jonesem, což byl další hlavní nepřítel farmy, symbolizoval Trockého (politický rival Stalina, pak byl donucen utéct)
           	\item[Boxer,] nebyl moc moudrý, ale chtěl, aby se on a vlastně všechna zvířata na farmě měla dobře, proto pracoval až do vyčerpání, myšlence animalismu věřil, chtěl jsen spravedlivou společnost a bezproblémový život, symbolizuje pracovitou dělnickou třídu
            \item[Major,] nastolil původní myšlenku,  revoluce, může být srovnáván s Leninem nebo Marxem, na rozdíl od ostatních prasat mu záleželo na té myšlence
            \item[Molina,] zajímala se o sebe, měla ty stužky, později odešla, představuje emigranty
            \item[Lupina,] taky symbolizuje dělnickou třídu, ale na rozdíl od Boxera uměla trochu číst, trochu o věcech pochybovala
            \item[ovce] symbolizují nejhloupější část populace, jen opakují určité fráze
            \item[Benjamin,] osel, pořád má narážky na to, že si něco pamatuje, symbolizuje inteligenci
            \item[Mojžíš,] havran, symbolizuje víru
        \end{description}
    \item[názor:] názor
    \item[kontext:]  Dělám si poctivé zápisky.
    \item[zdroje:] $ $
    \begin{itemize}
        \setlength\itemsep{0em}
        \item[$-$] BRADBURY, Ray. \textit{451 stupňů Fahrenheita} Online. ? ?, ? Dostupné z: \url{https://www.milujemecestinu.cz/citanka/Bradbury-Ray---451-stupňů-Fahrenheita.pdf}. [cit. 2025-03-2].
    \end{itemize}
\end{description}
\end{document}
