\documentclass{article}
\usepackage{fullpage}
\usepackage[czech]{babel}
\usepackage{amsfonts}
\usepackage{hyperref}

\title{\vspace{-2cm}Farma zvířat (George Orwell)\vspace{-2cm}}
\date{}
\author{}

\begin{document}
\maketitle
\section{První část.}
\begin{description}
    \setlength\itemsep{0.15em}
    \item[druh:] epika s lyrickými prvky
    \item[žánr:] antiutopický román
    \item[téma:] Přestože je někdy revoluce prováděna s dobrými úmysly, vládnoucí vrstva může zneužít své moci.
    \item[motivy:] prase, mlýn, dynamo, křída, dům, postel, čtení, přikázání, píseň Zvířata Anglie
    \item[zařazení výňatku do kontextu díla:]
    \item[časoprostor:] na Panské farmě v Anglii, pravděpodobně za Viktoriánské Anglie (na zdi tam byl obraz královny Viktorie)
    \item[kompoziční výstavba:] chronologická s retrospektivními odbočkami
\end{description}
\section{Druhá část.}
\begin{description}
    \setlength\itemsep{0.15em}
    \item[vypravěč:] nad dějem, nevyjadřuje vlastní názor, není jedna z postav
    \item[vyprávěcí způsoby:] \textit{--er forma}
    \item[typy promluv:] nepřímá řeč, přímá řeč, nevlastní přímá i polopřímá řeč
    \item[jazyková stránka:] spisovná čeština
    \item[postavy:]
        \begin{description}
            \setlength\itemsep{0.15em}
 			\item[Napoleon,] vůdce farmy, v podstatě ze sebe moudrým způsobem udělal diktátora, jeho jediným cílem je ovládnout ostatní zvířata a sým se mít dobře, je mu úplně jedno, jak si vedou ostatní
   	        \item[Pištík,] jeho přisluhovač / mluvčí, v podstatě stejného charakteru
           	\item[Kuliš,] nepřítel Napoleona, byl vyhnán a prohlášen za zrádce farmy, později se vlastně poszměňuje minulost, kdy o něm
                říkají, že se spolčil s Jonesem, což byl další hlavní nepřítel farmy
           	\item[Boxer,] nebyl moc moudrý, ale chtěl, aby se on a vlastně všechna zvířata na farmě měla dobře, proto pracoval až do vyčerpání, myšlence animalismu věřil, chtěl jsen spravedlivou společnost a bezproblémový život
        \end{description}
    \item[názor:] názor
    \item[kontext:]  Dělám si poctivé zápisky.
    \item[zdroje:] $ $
    \begin{itemize}
        \setlength\itemsep{0em}
        \item[$-$] BRADBURY, Ray. \textit{451 stupňů Fahrenheita} Online. ? ?, ? Dostupné z: \url{https://www.milujemecestinu.cz/citanka/Bradbury-Ray---451-stupňů-Fahrenheita.pdf}. [cit. 2025-03-2].
    \end{itemize}
\end{description}
\end{document}
