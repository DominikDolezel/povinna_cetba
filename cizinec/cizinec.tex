\documentclass{article}
\usepackage{fullpage}
\usepackage[czech]{babel}
\usepackage{amsfonts}
\usepackage{hyperref}

\title{\vspace{-2cm}Cizinec (Albert Camus)\vspace{-2cm}}
\date{}
\author{}

\begin{document}
\maketitle
\section{První část.}
\begin{description}
    \setlength\itemsep{0.15em}
    \item[druh:] próza (jasný děj, lyrická stránka je takřka neexistující)
    \item[žánr:] román, přestože krátký, ale je tam hlavní postava a my se postupně dozvídáme její osudy
    \item[téma:] zvláštní člověk, který nikoho nechápe, smrt jako jediné východisko jeho situace, jeho poctivost,
        smířenost s jeho osudem
    \item[motivy:] cigarety, vězení, šibenice, soud, písek, moře, autobus, smrt, matka, útulek (pro důchodce)
    \item[zařazení výňatku do kontextu díla:] z konce díla, kdy se hlavní hrdina hádá s Kaplanem o bohu???
    \item[časoprostor:] čas ani prostor blíže neurčen, pravděpodobně někdy z doby života autora, tedy první polovina
        20. století, údajně se odehrává v Alžírsku
    \item[kompoziční výstavba:] chronologická, členěno do dvou částí o 6 a 5 dalších částech, první je o samotném zločinu
        (jede na pohřeb matky, potká se s Marií, spřátelí se se sousedem Raymondem, jedou na pláž k jeho kamarádovi
        Massonovi, zastřelí Raymondova arabského pronásledovatele), druhá část se odehrává ve vězení a v soudní
        síni
\end{description}
\section{Druhá část.}
\begin{description}
    \setlength\itemsep{0.15em}
    \item[vypravěč:] vypravěčem je sama hlavní postava
    \item[vyprávěcí způsoby:] \textit{--ich forma}
    \item[typy promluv:] přímá řeč
    \item[jazyková stránka:] v podstatě soudobá čeština
    \item[postavy:]
        \begin{description}
            \setlength\itemsep{0.15em}
           	\item[Mersault,] hlavní postava, vypravěč, cizinec. Cizinec v komunitě Arabů, ale i v komunitě Francouzů, jejichž morální konvence
                neuznává. Je chytrý (chytřejší než ostatní vězni), ale je mu v podstatě všechno jedno. Na ničem mu moc nezáleží, ale to je
                jeho povahou. Na konci už v podstatě rezignuje na jakoukoliv obhajobu, protože poznává, že soud je veden bez jeho vstupů,
                ačkoliv on je obžalovaný.
            \item[Marie,] jeho slečna, která se do něj upřímně zamiluje a chtějí se vzít. Je hodná a Mersault se jí opravdu líbí, přes
                jeho podivínství.
            \item[Raymond,] Mersaultův soused a kamarád, je velice hrubý, chce vytrestat svoji bývalou přítelkyni, požádá Mersaulta o pomoc,
                on souhlasí, přestože ví, že dojde k fyzickému násilí na ženě.
            \item[Céleste,] majitel restaurace, do které Mersault chodí na jídlo. Při soudním líčení chce Mersaultovi pomoct, je srdečný.
        \end{description}
    \item[názor:] názor
    \item[kontext:]  Dělám si poctivé zápisky.
    \item[zdroje:] $ $
    \begin{itemize}
        \setlength\itemsep{0em}
        \item[$-$] CAMUS, Albert. \textit{Cizinec} Online. ? ?, ? Dostupné z: \url{https://www.gmct.cz/media/files/library/PDF/Cizinec%20-%20Albert%20Camus.pdf}. [cit. 2025-01-22].
    \end{itemize}
\end{description}
\end{document}
