\documentclass{article}
\usepackage{fullpage}
\usepackage[czech]{babel}
\usepackage{amsfonts}
\usepackage{hyperref}

\title{\vspace{-2cm}Petr a Lucie (Romain Rolland)\vspace{-2cm}}
\date{}
\author{}

\begin{document}
\maketitle
\section{První část.}
\begin{description}
    \setlength\itemsep{0.15em}
    \item[druh:] epika (jasně vnímatelný děj, i když někdy bývá upozaděn pocity, myšlenkami postav či popisem prostředí -- mnoho lyrických částí)
    \item[žánr:] novela (jediná dějová linka bez protáhlých odboček, v závěru nečekaný zvrat v podobě jejich smrti)
    \item[téma:] smysl války, života a lásky; milostný vztah dvou lidí, láska jako únik před negativním prostředím války, naivita lidí při dospívání, vyrovnávání se světem
    \item[motivy:] obrazy, mlha, metro, svačiny, lavička, kostel, válka, ostřelování, rusovlasá holčička
    \item[zařazení výňatku do kontextu díla:] těsně vedle nich vybuchuje bomba, ale oni si toho vůbec nevšímají, protože jsou na to už zvyklí; zároveň posunout svůj vztah do fyzické podoby, ale zároveň si ještě nějakou dobu užít to poznávání druhého člověka
    \item[časoprostor:] od středy večer 30. ledna do Velkého pátku 29. března 1918, Paříž
    \item[kompoziční výstavba:] děj chronologický s retrospektivními odbočkami, členěno do kapitol, nejsou však ani číslované
\end{description}
\section{Druhá část.}
\begin{description}
    \setlength\itemsep{0.15em}
    \item[vypravěč:] častokrát se ponořuje do myšlenek postav, popisuje jejich pocity apod., někdy i promlouvá k čtenářům, není tedy úplně nezaujatý
    \item[vyprávěcí způsoby:] hlavně \textit{--er} forma, v dialozích \textit{--ich} forma, objevuje se taky polopřímá řeč
    \item[typy promluv:] přímá a nepřímá řeč, myšlenky postav, polopřímá řeč
    \item[jazyková stránka:] jazyk spisovný, chvílemi zastaralejší výrazy (přechodníky, a to jak minulý, tak přítomný), francouzské výrazy
    \item[postavy:]
        \begin{description}
            \setlength\itemsep{0.15em}
            \item[Lucie,] jako obživu kreslí obrazy, hodná a laskavá, poctivá, trošku sobecká ve vztahu k mamince, protože si hledá nového manžela
            \item[Petr,] její snoubenec, trochu zženštělý, navzájem se nesmírně milují, jsou poctiví a nechápají ty, kteří dokážou zabíjet, oproti Lucii je trošku naivní ve vidění světa
            \item[Luciina matka,] dříve se měly rády, teď se však vzdálily, protože se matka znovu vdá a později otěhotní
            \item[Petrovi rodiče,] otec věrný státu, matka věrná křesťanka
            \item[Filip,] Petrův bratr, slouží na vojně, mají se s bratrem rádi, Petr ho má za jakýsi vzor, ke konci je však víc sebestředný a chce, aby se o něj Petr zajímal, ale ten myslí na Lucii, když je pak Filip náhodně uvidí v parku, dojde mu to a zase se udobří
        \end{description}
    \item[názor:] Přestože někomu může přijít trochu \uv{přeslazená}, mně se líbila, a hlavně dobře četla. 
    \item[kontext:] Vedu si poctivé zápisky.
    \item[zdroje:] $ $
    \begin{itemize}
        \setlength\itemsep{0em}
        \item[$-$] ROLLAND, Romain. \textit{Petr a Lucie} [online]. Městská knihovna v Praze, 2018 [cit. 2024-04-07]. Dostupné z: https://web2.mlp.cz/koweb/00/04/40/96/17/petr\_a\_lucie.pdf
    \end{itemize}
\end{description}
\end{document}
