\documentclass{article}
\usepackage{fullpage}
\usepackage[czech]{babel}
\usepackage{amsfonts}
\usepackage{hyperref}

\title{\vspace{-2cm}Pásmo (Guillaume Apollinaire)\vspace{-2cm}}
\date{}
\author{}

\begin{document}
\maketitle
\section{První část.}
\begin{description}
    \setlength\itemsep{0.15em}
    \item[druh:] lyrika (popisování různých životních situací, obrazů)
    \item[žánr:] pásmo (jak naznačuje už sám název, podle této básně se útvar později pojmenoval; volný tok vědomí, bez interpunkce, volný verš, spousta témat, malování obrazů, na které nahlíží z různých úhlů)
    \item[téma:] myšlenky na život (vitalismus), ženy, vzpomínání na cestování, opovrhování antikou a vším starým, oslava nových techologií, rychlost moderní společnosti (futurismus), cestování životem i prostorem, neměnnost a soudobost křesťanství, báseň lze rozdělit do jednotlivých částí:
    \begin{itemize}
      \item popisuje rychlost vývoje doby, jediné, co zůstává je křesťanství
      \item popisuje soudobou Paříž průmyslové i kancelářské práce, poté vzpomínka do dětství, opět silně akcentovaná víra
      \item oslavuje létání, ptáci, letadla, propojení, umožnění pokroku, přemísťování v čase
      \item popisuje hledání lásky, období dospívání, hledání životního partnera, chaos doby, přemísťování v prostoru
      \item popisuje Prahu, dopřeje si i čas na odpočinek, připomínání života
      \item vzpomíná na minulost, na svůj život, hovoří i o emigrantech, co odjíždí z Francie
      \item nachází se v hospodě s prostitutkami, je mu jich líto, ale zároveň si jich užívá, alkohol má zároveň rád a zároveň ne
    \end{itemize}
    \item[motivy:] alkohol, ženy a láska, Praha, Paříž, hospodská zahrádka, řeka, cestování, křesťanství, ryba, květiny, ulice, létání, ptáci
    \item[časoprostor:] střípky, jdou od antiky až po součanost, prostor: taky nejednoznačný, výrazné skoky časové i prostorové
    \item[kompoziční výstavba:] chronologická s restrospektivními odbočkami, ve verších, rozdělení do slok, zcela nepravidelné
\end{description}
\section{Druhá část.}
\begin{description}
    \setlength\itemsep{0.15em}
    \item[lyrický subjekt:] pravděpodobně sám Guillaume Apollinaire, hovoří o vlastních vzpomínkách, jsme v jeho vědomí, hovoří v druhé osobě k svému minulému já, částečně i \textit{--ich} forma a při popisu věcí \textit{--er} forma
    \item[vyprávěcí způsoby:] monolog, nejprve mluví ve třetí odobě, potom ve druhé, když se přesouvá do vzpomínek na dětství
    \item[typy promluv:] nejsou tam promluvy, jedná se o \textit{proud vědomí}, není to však přímá ani nepřímá řeč
    \item[jazyková stránka:] složitý jazyk, archaismy, přechodníky
    \item[postavy:] postavy se v díle nevyskytují, objevují se jen aluze na biblické či mýtické postavy a dále soudce a prostitutka
    \item[názor:] Pásmo mi přijde jako literární žánr naprosto cizí. Nejde se v něm orientovat. Je těžké pochopit, co tím chtěl autor vlastně říci. Asi modernistická ltieratura není určena pro mě.
    \item[kontext:] Vedu si podrobné zápisky.
    \item[zdroje:] $ $
    \begin{itemize}
        \setlength\itemsep{0em}
        \item[$-$] APOLLINAIRE, Guillaume. \textit{Pásmo} [online]. Městská knihovna v Praze, 2018 [cit. 2024-03-06].
    \end{itemize}
\end{description}
\end{document}
